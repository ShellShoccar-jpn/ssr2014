他の章では冒頭で「問題」を提示し、それに対する「回答」という形で役立つレシピを様々紹介しているが、この章ではそういった具体的な問題に取り組むのではなく、どのUNIX系OS上でも動くコードを記述するために都度確かめてておきたい事項を辞典形式でまとめた。

次の3項目から構成される。
\begin{quotation}
\noindent
\begin{description}
  \item[1 .文法・変数] シェル(Bourne Shell)自身が解釈する構文や変数について、どの環境でも使える書き方を記しているす。
  \item[2. 正規表現]   コマンドやその実装によって通用する正規表現のメタ文字の範囲が異なるため、それを体系的にまとめている。
  \item[3. コマンド]   POSIXの範囲で使えるコマンド、オプションはどの範囲かということをコマンドのアルファベット順にまとめている。
\end{description}
\end{quotation}
どの環境でも使えるシェルスクリプトを記述する際には本書を作業机に置いておき、確認したい項目をこまめに引くようにするとよいだろう。

尚、基本的にはPOSIX(``IEEE Std 1003.1'')に準拠させるという観点でまとめているが、一部例外もある。POSIXは様々なUNIX系OSが持つ仕様の共通部分を元に規定された経緯をもつが、OS間で相反するために採用できなかった仕様というものが一部存在し、現存するOSにもこれを残しているものがある。従ってPOSIXに準拠していれば、UNIX系OSで100\%{}通用するとは言えないため、そのような例外については個別に解説している。

またPOSIX文書には記されていない、現場で得た実戦的なノウハウも記してある。


\section*{■第一部 文法・変数など}

まずはシェル自身の文法や変数、多くのUNIXコマンドに共通する仕様などについて、どの環境でも動くようにするための注意点を記す。

\section{環境変数等の初期化}
\label{allenvs:initenv}

シェルスクリプトが起動した時、現時点で設定されている環境変数等に影響されて意図しない動作をするようでは、「どの環境でも動く」いう趣旨を満たしているとは言えない。従って環境変数等は一般的な内容に初期化しておくべきである。

次のコードをシェルスクリプトの冒頭に書いておくことをお勧めする。

\noindent
\begin{frameboxit}{160.0mm}
\begin{verbatim}
	set -u
	umask 0022
	PATH='/usr/bin:/bin'
	IFS=$(printf ' \t\n_'); IFS=${IFS%_}
	export IFS LC_ALL=C LANG=C PATH
\end{verbatim}
\end{frameboxit}
最初の行の\verb|set -u|は、環境変数等の初期化とは若干趣旨が異なるが、未定義の変数を読み出そうとした場合にエラー終了させるための宣言であり、strictなコードを記述したい場合に役立つだろう。

\section{シェル変数}

まず配列は使えない。従ってbashに存在する組込変数である\verb|PIPESTATUS|も使えない。同じことをがやりたいのなら\ref{recipe:Sayonara_PIPESTATUS}(PIPESTATUSさようなら)を参照してもらいたい。

変数の中身を部分的に取り出す記述に関して使っても大丈夫なものは、次のとおりである。

\begin{table}[H]
  \caption{POSIXで対応しているシェル変数の展開書式}
  \begin{center}
  \begin{tabular}{l!{\VLINE}>{\PBS\raggedright}m{38zw}}
    \HLINE
    書式 & 条件と動作 \\
    \hline
    \hline
      \verb|${var:-word}| & 変数\verb|$var|が未定義か空文字の場合は、文字列``\verb|word|''が読み出される。 \\
    \hline
      \verb|${var-word}|  & 変数\verb|$var|が未定義の場合は、文字列``\verb|word|''が読み出される。 \\
    \hline
      \verb|${var:=word}| & 変数\verb|$var|が未定義か空文字の場合は、文字列``\verb|word|''が読み出されると共に変数\verb|$var|にも代入する。 \\
    \hline
      \verb|${var=word}|  & 変数\verb|$var|が未定義の場合は、文字列``\verb|word|''が読み出されると共に変数\verb|$var|にも代入する。 \\
    \hline
      \verb|${var:?word}| & 変数\verb|$var|が未定義か空文字の場合は、文字列``\verb|word|''が読み出されると共に、エラー扱い(戻り値\verb|$?|が非ゼロ)にする。 \\
    \hline
      \verb|${var?word}|  & 変数\verb|$var|が未定義の場合は、文字列``\verb|word|''が読み出されると共に、エラー扱い(戻り値\verb|$?|が非ゼロ)にする。 \\
    \hline
      \verb|${var:+word}| & 変数\verb|$var|が未定義でも空文字でもなければ、文字列``\verb|word|''が読み出される。 \\
    \hline
      \verb|${var+word}|  & 変数\verb|$var|が未定義でなければ、文字列``\verb|word|''が読み出される。 \\
    \hline
      \verb|${#var}|      & 変数\verb|$var|の文字数(現在のロケールに応じた文字数)が読み出される。 \\
    \hline
      \verb|${var#pattern}|  & 変数\verb|$var|に格納されている文字列の左端に、パターン``\verb|pattern|''があれば、それが最小マッチングで切り落とされて読み出される。例えば\verb|var="/a/b/c"; echo "${var#*/}"|の場合、``\verb|a/b/c|''が読み出される。 \\
    \hline
      \verb|${var##pattern}| & 変数\verb|$var|に格納されている文字列の左端に、パターン``\verb|pattern|''があれば、それが最大マッチングで切り落とされて読み出される。例えば\verb|var="/a/b/c"; echo "${var##*/}"|の場合、``\verb|c|''が読み出される。 \\
    \hline
      \verb|${var%pattern}| & 変数\verb|$var|に格納されている文字列の右端に、パターン``\verb|pattern|''があれば、それが、最小マッチングで切り落とされて読み出される。例えば\verb|var="/a/b/c"; echo "${var%/*}"|の場合、``\verb|/a/b|''が読み出される。 \\
    \hline
      \verb|${var%%pattern}| & 変数\verb|$var|に格納されている文字列の右端に、パターン``\verb|pattern|''があれば、それが、最大マッチングで切り落とされて読み出される。例えば\verb|var="/a/b/c"; echo "${var%%/*}"|の場合、``\verb||''(空文字)が読み出される。 \\
    \HLINE
  \end{tabular}
  \label{tbl:signal_no}
  \end{center}
\end{table}

いずれも、条件に該当しない場合は元のシェル変数に格納されている文字列が読み出される。

\section{スコープ}

\noindent
→\ref{allenvs:local_keyword}(local修飾子)を参照

\section{正規表現}
\label{allenvs:regexp}

正規表現に関して注意しなければならないことは、コマンドの種類(AWK、grep、sedなど)や、また同じコマンドでも環境によって使えるメタ文字の範囲が異なるということだ。詳細に関しては本章\textbf{第二部(正規表現)}を参照してもらいたい。

ちなみに、「シェル変数の正規表現は?」と質問する人がいるかもしればいが、それは一部シェルの独自拡張機能なのでどの環境でも使えるものではない。

\noindent
→\ref{allenvs:locale}(ロケール)、\ref{allenvs:letterclass}(文字クラス)も参照

\section{文字クラス}
\label{allenvs:letterclass}

\verb|[[:alnum:]]|のように記述して使う「文字クラス」というものがある。だが、文字クラスは使わない方が無難だ。

これの正式名称は「POSIX文字クラス」\footnote{使えるもの一覧は、\verb|http://pubs.opengroup.org/onlinepubs/9699919799/basedefs/V1_chap09.html#tag_09_03_05|参照}という。その名のとおりPOSIX準拠であるのだが、Raspberry PiのAWKなど、一部の実装ではうまく動いてくれない。

まぁ、POSIXに準拠してないそっちの実装が悪いといってしまえばそれまでなのだが、そもそも設定されているロケールによって全角を受け付けたり受け付けなかったりして環境の影響を受けやすいので使わない方がよいだろう。

\noindent
→\ref{recipe:zenkaku_regexp}(全角文字に対する正規表現の扱い)も参照

\section{乱数}
\label{allenvs:random_number}

乱数を求めたい時、bashの組込変数\verb|RANDOM|を使うのは論外だが、「それなら」とAWKコマンドのrand関数とsrand関数を使えばいいやと思うかもしれないがちょっと待った!

論より証拠。FreeBSDで次の記述を何度も実行してみれば、非実用的であることがすぐわかる。

\begin{screen}
	\verb|$ for n in 1 2 3 4 5; do awk 'BEGIN{srand();print rand();}'; sleep 1; done| \return \\
	\verb|0.0205896| \\
	\verb|0.0205974| \\
	\verb|0.0206052| \\
	\verb|0.020613| \\
	\verb|0.0206209| \\
	\verb|$ |
\end{screen}

つまり動作環境によっては乱数としての質が非常に悪いのだ。AWKが内部で利用しているOS提供ライブラリ関数のrand()とsrand()を、FreeBSDは低品質だったオリジナルのまま残し、新たにrandom()という別の高品質乱数源関数を提供することで対応しているのが理由なのだが……。(Linuxではrand()とsrand()を内部的にrandom()にしている)

\subsection*{/dev/urandomを使うのが現実的}

ではどうすればいいか。POSIXで定義されているものではないが、\verb|/dev/urandom|を乱数源に使うのが現実的だと思う。例えば次のようにしてod、sedコマンドを組み合わせれば0$\sim$4294967295の範囲の乱数が得られる。

\begin{screen}
	\verb!$ od -A n -t u4 -N 4 /dev/urandom | sed 's/[^0-9]//g'! \return
\end{screen}

最後の段でtrコマンドではなくsedコマンドを使っている理由については、\ref{allenvs:tr}(trコマンド)を参照。

\subsection*{/dev/urandomをどうしても使いたくない場合}

乱数の品質は\verb|/dev/urandom|ほど高くないものの、代替手段はある。psコマンドの結果は実行するたびに必ず変化するのでこれを種として取り入れる。

具体的には、プロセスID、実行時間、CPU使用率、メモリ使用量の各一覧あたりが刻々と変化するのでこれらを取得するとよいだろう。更に、現在日時も加え、これらに基づいて$2^32$未満の範囲でAWKのsrand()に渡す乱数の種を生成しているのが次のコードだ。

\noindent
\begin{frameboxit}{160.0mm}
\begin{verbatim}
	LF=$(printf '\\\n_');LF=${LF%_}     # sedで改行を扱うための定義
	(ps -Ao pid,etime,pcpu,vsz; date) | # 乱数源(プロセス情報一覧+日時) 
	od -t d4 -A n -v                  | # 数値化する
	sed 's/[^0-9]\{1,\}/'"$LF"'/g'    |
	grep '[0-9]'                      |
	tail -n 42                        | # 100000000未満の数字を
	sed 's/.*\(.\{8\}\)$/\1/g'        | # 42個まで用意(2^32未満にするため)
	awk 'BEGIN{a=-2147483648;}        # # 上の値を足してsigned long値を作る
	     {a+=$1;}                     #
	     END{srand(a);print rand();}'
\end{verbatim}
\end{frameboxit}


\section{ロケール}
\label{allenvs:locale}

どの環境でも動くことを重視するなら、環境変数の中でもとりわけロケール系環境変数の内容には注意しなければならない。
理由は、ロケール環境変数(\verb|LANG|や\verb|LC_*|)の内容によって動作が変わるコマンドがあるからだ。

具体的に何が変わるかといえば、主に文字列長の解釈や、出力される日付である。下記にそれらをまとめてみた。

\subsection*{ロケール系環境変数の影響を受けるもの}

\subsubsection*{入力文字列の解釈が変わるもの}

例えば環境変数\verb|LANG|や\verb|LC_*|等の内容によって、全角文字を半角の相当文字と同一扱いしたり、全角文字の文字列長を1とするものとして、次のようなものがある。

\begin{itemize}
  \item AWKコマンド、grepコマンド、sedコマンド等の正規表現(\verb|[[:alnum:]]|、\verb|[[:blank:]]|等の文字クラスや、\verb|+|、\verb|\{n,m\}|などの文字数指定子)
  \item AWKコマンドの文字列操作関数(\verb|length|、\verb|substr|)
  \item wcコマンドの文字数(\verb|-m|オプション)
\end{itemize}

\noindent
など。

\subsubsection*{列区切り文字が変わるもの}

環境変数\verb|LANG|の内容によって、デフォルトの列区切り文字に全角スペースが加わるもの。

\begin{itemize}
  \item joinコマンド、sortコマンド等(\verb|-t|オプション)
\end{itemize}

\noindent
など。

\subsubsection*{出力フォーマットが変わるもの}

環境変数(\verb|LANG|や\verb|LC_*|)の内容によって、出力される文字列や書式が変わるもの。

\begin{itemize}
  \item dateコマンドのデフォルト日時フォーマット
  \item dfコマンドの1行目の列名の言語
  \item lsコマンド\verb|-l|オプションのタイムスタンプフォーマット
  \item シェルの各種エラーメッセージ
\end{itemize}

\noindent
など。

\subsubsection*{通貨や数値のフォーマットが変わるのも}

環境変数\verb|LC_MONETARY|や\verb|LC_NUMERIC|の影響を受けるもの。

\begin{itemize}
  \item sort……\verb|-n|オプションを指定した場合に、桁区切りのカンマの影響を受けたり受けなかったりする。
\end{itemize}

\subsection*{対策}

全ての環境で動くようにするのであれば、ロケール設定無しの状態、すなわち英語で使うべきであろう。対策方法を3つ紹介する。

\paragraph{envコマンドで全環境変数を無効化してコマンド実行}  \\
\begin{frameboxit}{160.0mm}
\begin{verbatim}
	echo 'ほげHOGE' | env -i awk '{print length($0)}'
\end{verbatim}
\end{frameboxit}

\paragraph*{\verb|LC\_{}ALL=C|及び\verb|LANG=C|を設定し、Cロケールにしてコマンド実行}  \\
\begin{frameboxit}{160.0mm}
\begin{verbatim}
	echo 'ほげHOGE' | LC_ALL=C LANG=C awk '{print length($0)}'
\end{verbatim}
\end{frameboxit}

\paragraph*{予め\verb|LC\_{}ALL=C|及び\verb|LANG=C|を設定しておく}  \\
\begin{frameboxit}{160.0mm}
\begin{verbatim}
	export LC_ALL=C LANG=C                     # シェルスクリプトの冒頭でこれを実行
	   :
	   :
	echo 'ほげHOGE' | awk '{print length($0)}' # そして目的のコマンドを実行
\end{verbatim}
\end{frameboxit}

ロケール系環境変数には現在、\verb|LANGUAGE|と\verb|LC_*|と\verb|LANG|がある。このうち各種\verb|LC_*|については\verb|LC_ALL|の設定によって全て上書きされるが、\verb|LANG|には効かないので\verb|LC_ALL|と\verb|LANG|を両方とも``\verb|C|''にする。最初に列挙した\verb|LANGUAGE|は最も強い効力を持つようだが、\verb|LANG|や\verb|LC_ALL|に``\verb|C|''が設定されている場合は無視されるということである\footnote{GNU gettextドキュメント2.3.3項 \\ \verb|http://www.gnu.org/software/gettext/manual/html_node/The-LANGUAGE-variable.html#The-LANGUAGE-variable|}。

ちなみに、いにしえの\verb|export|は、\verb|=|を使って変数の定義とexport化を同時に行えなかったということだが、今どきのPOSIXのmanページ\footnote{\verb|http://pubs.opengroup.org/onlinepubs/9699919799/utilities/V3_chap02.html#export|}によれば使えることになっている。
  \\

\noindent
→\ref{allenvs:initenv}(環境変数等の初期化)、\ref{recipe:sort}(sortコマンドの基本と応用とワナ)も参照

\section{\$((式))}

よく「計算をさせたければexprコマンドを使え」というが、今どきは\verb|$((式))|もPOSIXで規定されており、使っても問題無い。

ただ、数字の頭に``\verb|0|''や``\verb|0x|''を付けると、それぞれ8進数、16進数扱いされるのでexprコマンドとの間で移植をする場合は気を付けなければならない。(exprコマンドは、数字の先頭に``\verb|0|''が付いていても常に10進数と解釈される)

\begin{screen}
	\verb|$ echo $((10+10))| \return \verb|  |←10進数の10に、10進数の10を足す \\
	\verb|20| \\
	\verb|$ echo $((10+010))| \return \verb| |←10進数の10に、8進数の10を足す \\
	\verb|18| \\
	\verb|$ echo $((10+0x10))| \return ←10進数の10に、16進数の10を足す \\
	\verb|26| \\
	\verb|$ |
\end{screen}

この問題は、異なる実装のAWK間にもあるので注意。→\ref{allenvs:AWK}(AWKコマンド)参照

\section{case文}

\noindent
→\ref{allenvs:if}(if文)参照

\section{if文}
\label{allenvs:if}

たまに、elseの時は何かしたいけどthenの時は何もしたくないということがある。だからといってthenとelseの間に何も書かないと、bash等一部のシェルではエラーを起こしてしまう。

\paragraph{bashの場合、次のコードはエラーになる}  \\
\begin{frameboxit}{160.0mm}
\begin{verbatim}
	if [ -s /tmp/hoge.txt ]; then
	  # 1バイトでも中身があれば何もしない ←ここでエラー
	else
	  # 0バイトだったら消す
	  rm /tmp/hoge.txt
	fi
\end{verbatim}
\end{frameboxit}

elifの後もelseの後も同様であるし、case文でも条件分岐した先に何もコードを書いていなければ同じだ。要するに\textbf{bashでは、条件分岐先に有効なコードを置かないというコードが許されない}のだ。(コメントを書いただけではダメ)

\subsection*{対策}

何らかの無害な処理を書けばいいのだが、一番軽いのはnullコマンド(``\verb|:|'')ではないだろうか。つまり、こう書けばどの環境でも無難に動くようになる。

\paragraph{何もしたくなければnullコマンドを置くとよい(3行目に注目)}  \\
\begin{frameboxit}{160.0mm}
\begin{verbatim}
	if [ -s /tmp/hoge.txt ]; then
	  # 1バイトでも中身があれば何もしない ←今度はbashでもエラーにならない
	  :
	else
	  # 0バイトだったら消す
	  rm /tmp/hoge.txt
	fi
\end{verbatim}
\end{frameboxit}

別の対策としては、条件を反転してそもそもelse節を使わずに済むようにするのもいいだろう。しかしそれによってコードが読みにくくなったり、条件が3つ以上の複雑な場合などは、無理せずこの技法を用いるべきだ。

\section{local修飾子}
\label{allenvs:local_keyword}

シェル関数の中で用いる変数を、その関数内だけで有効なローカル変数にする場合に用いる修飾子だが、これはPOSIXでは規定されていない。しかし、関数内ローカルな変数は簡単に用意できる。小括弧で囲ってサブシェルを作ればその中で代入した値は外へは影響しないからだ。

次のシェル関数``localvar\_{}sample()''を見てもらいたい。中身を丸ごと小括弧で囲ったシェル関数で定義した次のシェル変数\verb|$a|、\verb|$b|、\verb|$c|は、関数終了後に消滅するし、外部に同名の変数があってもその値を壊すことはない。(だだし初期値はそれら外部変数の値になっている)

\paragraph{シェル関数内でローカルな変数を作る}  \\
\begin{frameboxit}{160.0mm}
\begin{verbatim}
	localvar_sample() {
	  (                   # ←小括弧で囲む
	    a=$(whoami)
	    b='My name is'
	    c=$(awk -v id=$a -F : '$1==id{print $5}' /etc/passwd)
	    echo "$b $c."
	  )
	}
\end{verbatim}
\end{frameboxit}

\section{PIPESTATUS変数}

例えば組込変数\verb|PIPESTATUS|に依存したシェルスクリプトが既にあって、それをどの環境でも使えるように書き直したいと思った場合、実は可能だ。詳しいやり方については、\ref{recipe:Sayonara_PIPESTATUS}(PIPESTATUSさようなら)を参照してもらいたい。

\section*{■第二部 正規表現}

正規表現は、コマンドによって使えるメタ文字の範囲が異なるうえ、実装によってはPOSIXの範囲で使えるメタ文字の他に独自のメタ文字を追加していることもある。おかげで、結局どのコマンドでどのメタ文字が使えるのか(使ってもよいのか)わからず混乱しがちだ。そこで、POSIXの範囲を意識しながら「どのUNIXコマンドでも使える正規表現」をまとめた。

\subsection*{知っておくべきメタ文字セットは3つ}

知っておくべきメタ文字セットは、次に示す2種類+1サブセットの3つだけである。
\begin{description}
  \item[1.] BRE(基本正規表現)メタ文字セット\footnote{原文はPOSIXの9.3節``Basic Regular Expression''参照。\\(\verb|http://pubs.opengroup.org/onlinepubs/9699919799/basedefs/V1_chap09.html#tag_09_03|)}
  \item[2.] ERE(拡張正規表現)メタ文字セット\footnote{原文はPOSIXの9.4節``Extended Regular Expression''参照。\\(\verb|http://pubs.opengroup.org/onlinepubs/9699919799/basedefs/V1_chap09.html#tag_09_04|)}
    \begin{description}
      \item[$2\prime$.] AWKのサブセット
     \end{description}
\end{description}
もちろん、これ以外にもGNU拡張正規表現メタ文字セットやPerl拡張正規表現メタ文字セット、JavaScript拡張正規表現メタ文字セットなどいくつかあるのだが、UNIXにおける「どの環境でも(=POSIXで)使える」という特長を持たせたいのであれば、それらを意識する必要はなく、この3つさえおさえておけばよい\footnote{例えPOSIXにこだわらないとしても、他のものは大抵EREの拡張になっているので、一旦EREを覚えておくと整理しやすいだろう。}。

\subsubsection*{各コマンドは、どのメタ文字セットに対応しているか}

知っておくべきメタ文字セットが3つあることがわかったところで、各コマンドがそれら3つのうちどれに対応しているかをまとめたのが次の表だ。この表を見て、対応しているメタ文字セットがわかったら、次項のメタ文字セット各一覧を見ればよい。
\begin{table}[htb]
  \caption{各コマンドが対応しているメタ文字セット}
  \begin{center}
  \begin{tabular}{ll}
    \HLINE
    コマンド   & 対応しているメタ文字セット \\
    \hline
    AWK                 & EREのAWKサブセット \\
    ed                  & BREメタ文字セット  \\
    egrep               & BREメタ文字セット  \\
    ex                  & BREメタ文字セット  \\
    grep(\verb|-E|なし) & BREメタ文字セット  \\
    grep(\verb|-E|あり) & EREメタ文字セット  \\
    more                & BREメタ文字セット  \\
    sed                 & BREメタ文字セット  \\
    vi                  & BREメタ文字セット  \\
    \HLINE
  \end{tabular}
  \label{tbl:signal_no}
  \end{center}
\end{table}

参考までに述べておくと、\textbf{GNU拡張やPerl拡張、JavaScript拡張は、いずれもEREのスーパーセット}である。ということはすなわち、EREメタ文字セットを覚えておけばそれらの上でも動くということだ。

また、多くのgrep(\verb|-E|オプションなし)では、一部のEREメタ文字がバックスラッシュ付きで使えたりする(例えば``\verb|\+|''や``\verb!\|!'')が、それらはGNU拡張であってgrep本来のものではない。

目的のコマンドがどのメタ文字セットに対応しているかわかったところで、次項より各メタ文字セットを紹介する。

\subsection*{1. BRE(基本正規表現)メタ文字セット}

(BREに限らないが)メタ文字セットの中は、メタ文字を使う場所に応じてさらに3つのグループに分類される。それを踏まえて読んでもらいたい。

\subsubsection*{a-1. マッチを掛ける文字列(置換前文字列)のメタ文字一覧(ブラケット外部)}
まずはブラケット(\verb![!〜\verb!]!)の外部についてのみまとめる。ブラケット内部ではここで記す多くのメタ文字が意味を失ったり、あるいは意味が変わったり、ブラケット内部でのみ意味を持つメタ文字が新たに登場するため、次の表でまとめることにする。
\begin{table}[H]
  \begin{center}
  \begin{tabular}{l!{\VLINE}>{\PBS\raggedright}m{42zw}}
    \HLINE
        メタ文字 & 意味 \\
    \hline
    \hline
        \verb!^! & 文字列(通常は行)の先頭にマッチ(先頭以外では通常文字と見なされる) \\
    \hline
        \verb!$! & 文字列(通常は行)の末尾にマッチ(末尾以外では通常文字と見なされる) \\
    \hline
        \verb![!…\verb!]! & \verb![!と\verb!]!で囲まれた中で列挙した文字のいずれか1文字にマッチ \\
    \hline
        \verb![^!…\verb!]! & \verb![^!と\verb!]!で囲まれた中で列挙した文字\textbf{以外}の任意の1文字にマッチ \\
    \hline
        \verb!*! & \textbf{【繰返し指定子】} 直前に記述した文字が0文字以上連続していることを指定し、後続の繰返し指定子よりも優先して可能な限り最大数マッチさせようとする。 \\
    \hline
        \verb!\{!$n$\verb!\}! & \textbf{【繰返し指定子】} 直前に記述した文字が$n$文字連続していることを指定する。 \\
    \hline
        \verb!\{!$n$\verb!,\}! & \textbf{【繰返し指定子】} 直前に記述した文字が$n$文字以上連続していることを指定し、後続の繰返し指定子よりも優先して可能な限り最大数マッチさせようとする。 \\
    \hline
        \verb!\{!$m$\verb!,!$n$\verb!\}! & \textbf{【繰返し指定子】} 直前に記述した文字が$m$文字以上、$n$文字以下連続していることを指定し、後続の繰返し指定子よりも優先して可能な限り最大数マッチさせようとする。 \\
    \hline
        \verb!\(!…\verb!\)! & \textbf{【包括指定子】} \verb!\(!と\verb!\)!で囲まれた範囲の文字列を、上記の繰返し指定子の1文字として扱わせたい場合、もしくはsed等で置換後に再利用したい文字列範囲を指定したい場合に用いる。 \\
    \hline
        \verb!\!$n$ & \textbf{【後方参照子】} $n$番目に記した包括指定子でマッチした文字列にマッチする。例えば、\verb!ABC123ABCABC!という文字列を\verb!^\([A-Z]*\)123\1*$!という正規表現文字列に掛ければ、\verb!\1!は\verb!ABC!という文字列と見なされるため、この場合\verb!\1*!は末尾にある2つの\verb!ABC!にマッチする。 \\
    \hline
        \verb!\!$x$ & 上記のうちでバックスラッシュで始まらないメタ文字自身、あるいはAWKやsedなどで正規表現の始まりを示すために用いた文字自身を指定したい場合に、$x$の部分にその文字を記述すればそれにマッチ \\
    \hline
        \verb!\\! & バックスラッシュ(\verb!\!)自身にマッチ \\
    \HLINE
  \end{tabular}
  \label{tbl:BRE_a1}
  \end{center}
\end{table}

\subsubsection*{a-2. マッチを掛ける文字列(置換前文字列)のメタ文字一覧(ブラケット内部)}
既に述べたように、ブラケット(\verb![!〜\verb!]!)で囲まれた区間は外側とは使えるメタ文字が異なり、次の表のとおりである。
\begin{table}[H]
  \begin{center}
  \begin{tabular}{l!{\VLINE}>{\PBS\raggedright}m{42zw}}
    \HLINE
        メタ文字 & 意味 \\
    \hline
    \hline
        \verb!^! & 開きブラケットの直後(``\verb!-!''-や``\verb!]!''自身を指定したい場合でもそれらより手前)に記述すると、否定の意味になる。 \\
    \hline
        \verb!-! & 文字を列挙する代わりに範囲で指定できる。例えば\verb!A-Z!ならば文字\verb!A!から\verb!Z!を全て列挙したことと等価。もし``\verb!-!''自身を指定したい場合は、閉じブラケットの直前(閉じブラケット``\verb![!''自身も指定したい場合はそちらよりも後ろ)に記述する。 \\
    \hline
        \verb![! & 閉じブラケット(\verb!]!)の直前(だだし``\verb!-!''自身も指定する場合はそちらの方が後ろ)に記述すると``\verb![!''自身を指定することができる。 \\
    \hline
        \verb!]! & 開きブラケット(\verb![!)の直後に記述すると``\verb!]!''自身を指定することができる。 \\
    \hline
        \verb!\!$x$ & 上記のうちでバックスラッシュで始まらないメタ文字自身、あるいはAWKやsedなどで正規表現の始まりを示すために用いた文字自身を指定したい場合に、$x$の部分にその文字を記述すればそれにマッチ \\
    \hline
        \verb!\\! & バックスラッシュ(\verb!\!)自身にマッチ \\
    \HLINE
  \end{tabular}
  \label{tbl:BRE_a1}
  \end{center}
\end{table}

実はBRE(後述のEREも含む)では、上記に加えて次の表に示すメタ文字列が定義されているのだが、間違った実装がされていたり、使える実装にお目にかかったことがなかったり、といったものであるため、使うことはお勧めできない。

\begin{table}[H]
  \begin{center}
  \begin{tabular}{l!{\VLINE}>{\PBS\raggedright}m{42zw}}
    \HLINE
        メタ文字 & 意味 \\
    \hline
    \hline
        \verb![:!\textit{word}\verb!:]! & \textbf{【POSIX文字クラス】} \textit{word}の部分には\verb!alnum!(アルファベットと数字全部)、\verb!cntrl!(制御文字全部)、\verb!lower!(アルファベット小文字全部)、\verb!space!(スペースとタブと改ページ)、\verb!alpha!(アルファベット全部)、\verb!!digit!(数字全部)、\verb!print!(制御文字以外の文字全部)、\verb!upper!(アルファベット大文字全部)、\verb!blank!(スペースとタブ)、\verb!graph!(制御文字とスペース、タブ以外全部)、\verb!punct!(句読点全部)、\verb!xdigit!(16進数文字全部)が指定できる。実際に使うときは\verb![[:lower:][:blank:]]!などのように使う。しかし、一部の実装ではブラケット記号が一重でないと動かないといった間違った実装になっているものがある。 \\
    \hline
        \verb![.!\textit{word}\verb!.]! & 例えば\verb![[.hoge.]]!と記述したら、\verb!\(hoge\)\{1,\}!と等価な意味を持つようだ。しかし、使える実装を見たことがない。後者の記述で事足りるからだろうか? \\
    \hline
        \verb![=!$x$\verb!=]! & 例えば\verb![=a=]!と記述したら、``\verb!a!''にも``$\grave{a}$''にも``$\hat{a}$''にもマッチするもので、実際に使う時は\verb![[=a=]bc]!のように記述する。だがこれも、使える実装を見たことがない。アクセント記号付の文字を素直に書き並べれば済むので無くても事足りるからだろうか? \\
    \HLINE
  \end{tabular}
  \label{tbl:BRE_a1}
  \end{center}
\end{table}

\subsubsection*{b. 置換後の文字列指定(sed等の``\verb!s/!\textit{A}\verb!/!\textit{B}\verb!/!''における\textit{B}の部分)で使えるメタ文字一覧}

正規表現は、マッチする文字列を検索するためだけでなく、マッチしたその文字列を加工(置換)するためにも用いられる。sedコマンドにおける``\verb|s/|\textit{A}\verb|/|\textit{B}\verb|/|''はそのための代表的な書式であるが、この\textit{B}の部分で使えるメタ文字を次の表に示す。

\begin{table}[H]
  \begin{center}
  \begin{tabular}{l!{\VLINE}>{\PBS\raggedright}m{42zw}}
    \HLINE
        メタ文字 & 意味 \\
    \hline
    \hline
        \verb!\!$n$ & $n$番目に記した包括指定子(\verb!\(!…\verb!\)!)で囲まれた範囲にマッチした文字列に置き換えられる。 \\
    \hline
        \verb!&! & マッチした文字列全体に置き換えられる。 \\
    \hline
        \verb!\!$x$ & 上記のメタ文字(\verb!&!)、またsed等で正規表現の始まりを示すために用いた文字自身を指定したい場合、$x$の部分に記せばその文字自身を指定することができる。 \\
    \hline
        \verb!\\! & バックスラッシュ(\verb!\!)自身を指定したい場合に用いる。 \\
    \HLINE
  \end{tabular}
  \label{tbl:BRE_a1}
  \end{center}
\end{table}

\subsection*{2. ERE(拡張正規表現)メタ文字セット}

EREはほぼBREを拡張したものになっているが、純粋の上位互換ではないので注意。EREのメタ文字一覧も表で示すが、予め違いを簡単に列挙すると次のとおりである。
\begin{itemize}
  \item 使えるメタ文字を追加。(``\verb!+!''、``\verb!?!''、``\verb!|!''))
  \item \textbf{【非互換】} バックスラッシュでエスケープしていた括弧類(``\verb!\(!''、``\verb!\)!''、``\verb!\{!''、``\verb!\}!'')がバックスラッシュ不要に。
  \item \textbf{【非互換】} 後方参照(``\verb!\!$n$'')が無保証に。(実際に使えない実装がある)
\end{itemize}

使えるメタ文字が増えている(+、?、|)が、純粋な上位互換ではないので注意。具体的には、バックスラッシュでエスケープしていた括弧類のメタ文字(\(、\)、\{、\})がバックスラッシュ不要になっている点、そして後方参照が保証されていない(実際に使えない実装がある)点である。

\subsubsection*{a-1. マッチを掛ける文字列(置換前文字列)のメタ文字一覧(ブラケット外部)}
BREと同様に、まずマッチを掛ける文字列(置換前文字列)として指定できるメタ文字をまとめると、次の表のとおりになる。
\begin{table}[H]
  \begin{center}
  \begin{tabular}{l!{\VLINE}>{\PBS\raggedright}m{42zw}}
    \HLINE
        メタ文字 & 意味 \\
    \hline
    \hline
        \verb!^! & 文字列(通常は行)の先頭にマッチ(先頭以外では通常文字と見なされる) \\
    \hline
        \verb!$! & 文字列(通常は行)の末尾にマッチ(末尾以外では通常文字と見なされる) \\
    \hline
        \verb![!…\verb!]! & \verb![!と\verb!]!で囲まれた中で列挙した文字のいずれか1文字にマッチ \\
    \hline
        \verb![^!…\verb!]! & \verb![^!と\verb!]!で囲まれた中で列挙した文字\textbf{以外}の任意の1文字にマッチ \\
    \hline
        \verb!*! & \textbf{【繰返し指定子】} 直前に記述した文字が0文字以上連続していることを指定し、後続の繰返し指定子よりも優先して可能な限り最大数マッチさせようとする。 \\
    \hline
        \verb!+! & \textbf{【繰返し指定子】} 直前に記述した文字が1文字以上連続していることを指定し、後続の繰返し指定子よりも優先して可能な限り最大数マッチさせようとする。 \\
    \hline
        \verb!{!$n$\verb!}! & \textbf{【繰返し指定子】} 直前に記述した文字が$n$文字連続していることを指定する。 \\
    \hline
        \verb!{!$n$\verb!,}! & \textbf{【繰返し指定子】} 直前に記述した文字が$n$文字以上連続していることを指定し、後続の繰返し指定子よりも優先して可能な限り最大数マッチさせようとする。 \\
    \hline
        \verb!{!$m$\verb!,!$n$\verb!}! & \textbf{【繰返し指定子】} 直前に記述した文字が$m$文字以上、$n$文字以下連続していることを指定し、後続の繰返し指定子よりも優先して可能な限り最大数マッチさせようとする。 \\
    \hline
        \verb!?! & \textbf{【繰返し指定子】} 直前に記述した文字が0文字以上1文字以下連続していることを指定する。 \\
    \hline
        \verb!(!…\verb!)! & \textbf{【包括指定子】} \verb!(!と\verb!)!で囲まれた範囲の文字列を、上記の繰返し指定子の1文字として扱わせたい場合、もしくはsed等で置換後に再利用したい文字列範囲を指定したい場合に用いる。または、後述の論理和指定子の範囲を限定したい場合に用いる。 \\
    \hline
        \verb!|! & \textbf{【論理和指定子】} この指定子の左の文字列または右の文字列でマッチさせることを指定する。左右の範囲は、前述の包括指定子の中であればその始端または終端まで、無ければ正規表現文字列全体の始端または終端まで(``\verb!^!''や``\verb!$!''をも内包させる)と見なされる。例えば\verb!^ABC|DEF$!は、\verb!^(ABC|DEF)$!ではなく\verb!(^ABC)|(DEF$)!の意味に解釈される。 \\
    \hline
        \verb!\!$x$ & 上記のうちでバックスラッシュで始まらないメタ文字自身、あるいはAWKやsedなどで正規表現の始まりを示すために用いた文字自身を指定したい場合に、$x$の部分にその文字を記述すればそれにマッチ \\
    \hline
        \verb!\\! & バックスラッシュ(\verb!\!)自身にマッチ \\
    \HLINE
  \end{tabular}
  \label{tbl:BRE_a1}
  \end{center}
\end{table}

\subsubsection*{a-2. マッチを掛ける文字列(置換前文字列)のメタ文字一覧(ブラケット内部)}
これはBREと同じ。

\subsubsection*{b. 置換後の文字列指定で使えるメタ文字一覧}
これもBREと同じ。たが置換後文字列を指定できるコマンドでEREに対応しているものは、POSIXの範囲では存在しない。(AWKは後述するのでここでは除く)

\subsection*{$2\prime$. AWKで使えるメタ文字セット}

AWKは基本的にはEREのメタ文字セットに対応しているが、残念なことにブレース(``\verb!{!''、``\verb!}!'')には対応していない。従って\textbf{ブレースによる繰返し指定はできず、大きな弱点}になっている。

一応2008年版のPOSIXではこれも含めてEREに完全対応するように勧告されたようだが、まだ年数が浅いために現存するAWK実装で対応しているものは少なく、実質的に完全なEREは通用しない。

またAWKでは、正規表現を制御する各構文や関数に正規表現文字列が渡される前に、AWK言語としてのエスケープ処理がなされるので注意が必要だ。具体的には、次のとおりである。

\begin{table}[H]
  \begin{center}
  \begin{tabular}{l!{\VLINE}>{\PBS\raggedright}m{42zw}}
    \HLINE
        文字 & 何の文字に置換されるか \\
    \hline
    \hline
        \verb!\\! & バックスラッシュ\verb|\|に置換される。 \\
    \hline
        \verb!\/! & スラッシュ\verb|/|に置換される。 \\
    \hline
        \verb!\"! & ダブルクォーテーション\verb|"|に置換される。 \\
    \hline
        \verb!\!\textit{ddd} & \textit{ddd}が3桁の8進数である時、その値の文字コードに該当する文字に置換される。 \\
    \hline
        \verb!\a! & ビープ音(BEL:文字コード0x07)に置換される。 \\
    \hline
        \verb!\b! & バックスペース(BS:文字コード0x08)に置換される。 \\
    \hline
        \verb!\f! & 改ページ(FF:文字コード0x0c)に置換される。 \\
    \hline
        \verb!\n! & 改行(LF:文字コード0x0a)に置換される。 \\
    \hline
        \verb!\r! & 行頭復帰(CR:文字コード0x0d)に置換される。 \\
    \hline
        \verb!\t! & 水平タブ(HT:文字コード0x09)に置換される。 \\
    \hline
        \verb!\v! & 垂直タブ(VT:文字コード0x0b)に置換される。 \\
    \hline
        \verb!\!(上記以外) & 未定義(通常は単にバックスラッシュを除いた文字に置換される) \\
    \HLINE
  \end{tabular}
  \label{tbl:AWK_escape}
  \end{center}
\end{table}

\subsubsection*{a-1. マッチを掛ける文字列(置換前文字列)のメタ文字一覧(ブラケット外部)}
ブレースを用いた繰返し指定子(``\verb!{!''、``\verb!}!'')以外の全てのEREメタ文字セットに対応している。ただし、バックスラッシュで始まる文字列を与えると先程の表のとおりのエスケープ処理を受ける。

\subsubsection*{a-2. マッチを掛ける文字列(置換前文字列)のメタ文字一覧(ブラケット内部)}
BREと同じ。ただし、バックスラッシュで始まる文字列を与えると先程の表のとおりのエスケープ処理を受ける。

\subsubsection*{b. 置換後の文字列指定(sub,gsub関数の第2引数)で使えるメタ文字一覧}
これも基本的にはBREと同じなのだが、次の2点に注意しなければならない。
\begin{itemize}
  \item ``\verb|\|$n$''($n$は自然数)というメタ文字には対応していない(これに対応しているのはGNU AWKで追加されたgensubという関数である)。
  \item 先程の表のとおりのエスケープ処理を受ける。このため、特に注意が必要なのは、\textbf{メタ文字として予約されている}``\verb|&|''\textbf{自身を指定したい場合}である。具体的には、ダブルクォーテーションの内側で``\verb|\&|''と書きたいのであれば、バックスラッシュ``\verb|\|''がエスケープされないように``\verb|\\&|''と記されなければならない。
\end{itemize}


\section*{■第三部 コマンド}

いくらシェルスクリプトの文法、また正規表現に気をつけても、呼び出すコマンドが一部の環境でしか通用しないような使い方では意味をなさない。次に、POSIXの範囲で使用可能なオプションや、同じ使い方をしているにも関わらず生じる動作の違い、そして対策、といったことを中心に、どの環境でも通用するコマンド使用法の各論を紹介する。

\section{``[''コマンド}

\noindent
→\ref{allenvs:test}(testコマンド)参照

\section{AWKコマンド}
\label{allenvs:AWK}

AWKはそれが1つの言語でもあるので、説明しておくべきことがたくさんある。

\subsection*{\verb|-0|(マイナス・ゼロ)}

FreeBSD 9.xに標準で入っているAWKでは、\verb|-1*0|を計算すると``\verb|-0|''という結果になる。

\paragraph{FreeBSD 9.1で$-1*0$を計算させてみると} \\
\begin{screen}
	\verb|$ awk 'BEGIN{print -1*0}'| \return \\
	\verb|-0| \\
	\verb|$ |
\end{screen}

ところがこの挙動は同じFreeBSDでも10.xでは確認されないし、GNU版AWKでも起こらないようだ。

このようにして、同じ0であっても``\verb|-0|''という二文字で返してくる場合のある実装もあるので注意してもらいたい。

\subsubsection*{マイナスを取り去るには}

計算で得られた結果に0を足せばよいようだ。
\begin{screen}
	\verb|$ awk 'BEGIN{print -1*0+0}'| \return \\
	\verb|0| \\
	\verb|$ |
\end{screen}

\subsection*{0始まり即値の解釈の違い}

頭に0が付いている数値を即値(プログラムに直接書き入れる値)として与えると、それを8進数と解釈するAWK実装もあれば10進数と解釈するAWK実装もある。

\paragraph{FreeBSDのAWKで即値の010を解釈させた場合} \\
\begin{screen}
	\verb|$ awk 'BEGIN{print 010;}'| \return \\
	\verb|10| \\
	\verb|$ |
\end{screen}

\paragraph{GNU版AWKで即値の010を解釈させた場合} \\
\begin{screen}
	\verb|$ awk 'BEGIN{print 010;}'| \return \\
	\verb|8| \\
	\verb|$ |
\end{screen}

どこでも同じ動きにしたければ文字列として渡せばよい。すると10進数扱いになる。

\paragraph{GNU版AWKでも文字列として"010"を渡せば10進数扱いされる}  \\
\begin{screen}
	\verb|$ awk 'BEGIN{print "010"*1;}'| \return \\
	\verb|10| \\
	\verb!$ echo 010 | awk 'BEGIN{print $1*1;}'! \return \\
	\verb|10| \\
	\verb|$ |
\end{screen}

\subsection*{length関数の機能制限}

大抵のAWK実装は、

\begin{screen}
	\verb|$ awk 'BEGIN{split("a b c",chr); print length(chr);}'| \return \\
	\verb|3| \\
	\verb|$ |
\end{screen}

とやると、きちんと要素数を返すだろう。しかし実装によってはこれに対応しておらず、エラー終了してしまうものがある。このため、例えば次ようにユーザー関数\verb|arlen()|を作り、配列の要素数はその\verb|arlen()|で数えるようにすべきだ。

\paragraph{配列の要素数を数える関数を自作しておく}  \\
\begin{frameboxit}{160.0mm}
\begin{verbatim}
	awk '
	  BEGIN{split("a b c",chr); print arlen(chr);}
	  function arlen(ar,i,l){for(i in ar){l++;}return l;}
	'
\end{verbatim}
\end{frameboxit}

幸い、\textbf{AWKの配列変数は参照渡し}なので要素の中身が膨大だとしてもそれは影響しない。(要素数が大きい場合はやはり負担がかかると思うのだが……)

\subsubsection*{length()が使えるなら使いたい}

「length()が使えるなら使いたい!」というワガママなアナタは、こうすればいい。

\paragraph{length()が使えるなら使いたいワガママなアナタへ}  \\
\begin{frameboxit}{160.0mm}
\begin{verbatim}
	# シェルスクリプトの冒頭で、配列に対してlength()を使ってもエラーにならないことを確認
	if awk 'BEGIN{a[1]=1;b=length(a)}' 2>/dev/null; then
	  arlen='length'  # ←エラーにならないならlength()
	else
	  arlen='arlen'   # ←エラーになるなら独自関数"arlen"
	fi

	awk '
	  BEGIN{split("a b c",chr); print '$arlen'(chr);}       # ←判定結果に応じて適宜選択される
	  function arlen(ar,i,l){for(i in ar){l++;}return l;}
	'
\end{verbatim}
\end{frameboxit}

\subsection*{printf、sprintf関数}

\noindent
→\ref{allenvs:printf}(printfコマンド)参照

\subsection*{rand関数,srand関数は使うべきではない}

\noindent
→\ref{allenvs:random_number}(乱数)参照

\subsection*{gensub関数は使えない}

GNU版AWKには独自拡張がいくつかあるが、中でも注意すべき点はgensub関数がその一つであることだ。互換性を優先するなら、多少不便かもしれないがsub関数やgsub関数を使こと。その他、こまごまと気を付けるべきことについては、GNU AWKマニュアル``\verb|--posix|''オプションに関する記述\footnote{\verb|http://www.gnu.org/software/gawk/manual/gawk.html#index-gawk_002c-extensions_002c-disabling|}が参考になる。

\subsection*{正規表現では有限複数個の繰り返し指定ができない}

AWKの正規表現は繰り返し指定が苦手。文字数指定子のうち、``\verb|?|''(0$\sim$1個)と``\verb|*|''(0個以上)と``\verb|+|''(1個以上)は使えるが、2個以上の任意の数を指定するための``\verb|{数}|''には対応していない。GNU版AWKでは独自拡張して使えるようになっているのだが……。

AWKで使える正規表現のメタ文字に関しては、本章第二部(正規表現)で詳しく紹介しているので参照してもらいたい。

\subsection*{整数の範囲}

例えば、あなたの環境のAWKは次のように表示されはしないだろうか?

\begin{screen}
	\verb|$ awk 'BEGIN{print 2147483648}'| \return \\
	\verb|2.14748e+09| 
\end{screen}

上記の例は、$\mathrm{0x7FFFFFFF}$(符号付き4バイト整数の最大値)より大きい整数を扱えないAWK実装である。このようなことがあるので、桁数の大きな数字を扱わせようとする時は注意が必要だ。計算をせず、単に表示させたいだけなら文字列として扱えばよい。

\subsection*{ロケール}

\noindent
→\ref{allenvs:locale}(ロケール)を参照

\section{dateコマンド}

元々の機能が物足りないがゆえか、各環境で独自拡張されているコマンドの一つだ。だが互換性を考えるなら、使えるのは

\begin{itemize}
  \item \verb|-u|オプション(=UTC日時で表示)
  \item ``\verb|+フォーマット文字列|''にて表示形式を指定
\end{itemize}
の2つだけと考えるが無難だろう。尚、フォーマット文字列中に指定できるマクロ文字の一覧は、POSIXのdateコマンドのmanページ\footnote{\verb|http://pubs.opengroup.org/onlinepubs/9699919799/utilities/date.html|}の``Conversion Specifications''の段落にまとめられているので参照されたい。

\subsection*{UNIX時間との相互変換}

マクロの種類はいろいろあるのだが、残念ながらUNIX時間\footnote{エポック秒とも呼ばれる``UTC 1970/1/1 00:00:00''からの秒数}との相互変換は無い。これさえできれば何とでもなるのだが……。

しかしこんなこともあろうかと、相互変換を行うコマンドを作ったのだ。もちろんシェルスクリプト製である。詳しくは、\ref{recipe:utconv}(シェルスクリプトで時間計算を一人前にこなす)を参照してもらいたい。

\section{duコマンド}

特定ディレクトリー以下のデータサイズを求めるこのコマンド、POSIXで規定されているオプションではないが\verb|-h|というものがある。これはファイルやディレクトリーのデータサイズをk(キロ)、M(メガ)、G(ギガ)等最適な単位を選択して表示するものだ。

しかしこのオプションの表示フォーマットは、環境によって僅かに異なる。

\paragraph{FreeBSDのduコマンド-hオプションの挙動}  \\
\begin{screen}
	\verb!$ du -h /etc | head -n 10! \return \\
	\verb|118K    /etc/defaults| \\
	\verb|2.0K    /etc/X11| \\
	\verb|372K    /etc/rc.d| \\
	\verb|4.0K    /etc/gnats| \\
	\verb|6.0K    /etc/gss| \\
	\verb| 30K    /etc/security| \\
	\verb| 40K    /etc/pam.d| \\
	\verb|4.0K    /etc/ppp| \\
	\verb|2.0K    /etc/skel| \\
	\verb|144K    /etc/ssh| \\
	\verb|$ |
\end{screen}

\paragraph{Linuxのduコマンド-hオプションの挙動}  \\
\begin{screen}
	\verb!$ du -h /etc | head -n 10! \return \\
	\verb|112K    /etc/bash_completion.d| \\
	\verb|12K     /etc/abrt/plugins| \\
	\verb|4.0K    /etc/statetab.d| \\
	\verb|4.0K    /etc/dracut.conf.d| \\
	\verb|28K     /etc/cron.daily| \\
	\verb|4.0K    /etc/audisp| \\
	\verb|4.0K    /etc/udev/makedev.d| \\
	\verb|36K     /etc/udev/rules.d| \\
	\verb|48K     /etc/udev| \\
	\verb|8.0K    /etc/sasl2| \\
	\verb|$ |
\end{screen}

違いがわかるだろうか? 1列目(サイズ)が、前者は右揃えなのに後者は左揃えなのだ。従ってどちらの環境でも動くようにするには、1列目であっても行頭にスペースが入る可能性を考慮しなければならない。

例えば1列目の最後に単位"B"を付加したいとしたら、下記の1行目はダメで、2行目の記述が正しい。

\paragraph{1行目の最後に"B"(単位)を付けたい場合}  \\
\begin{frameboxit}{160.0mm}
\begin{verbatim}
	du -h /etc | sed 's/^[0-9.]\{1,\}[kA-Z]/&B/'   # ←これでは不完全

	du -h /etc | sed 's/^ *[0-9.]\{1,\}[kA-Z]/&B/' # ←こうするのが正しい

	du -h /etc | awk '{$1=$1 "B";print}'           # ←折角の桁揃えがなくなるがまぁアリ
\end{verbatim}
\end{frameboxit}

このようにして1列目にインデントが入るコマンドは結構あるし、インデントの幅も環境によりまちまちなので注意が必要だ。(例、 \verb|uniq -c| 、\verb|wc|などなど)

\section{echoコマンド}

結論から言うと、どこでも動くよううにしたい場合、次の項目に1つでも当てはまるものには\textbf{echoコマンドは使うべきではない。}
\begin{itemize}
  \item 先頭がハイフンで始まる可能性がある文字列
  \item エスケープシーケンスを含む可能性のある文字列
\end{itemize}

理由は次のとおりである。

\subsection*{対応しているオプションが異なる}

例えば、Linuxのechoコマンドは\verb|-e|、\verb|-n|オプションに対応しており、第一引数に指定すれば、それを表示はせずにオプション文字列として解釈する。一方、FreeBSDのechoコマンド(外部コマンド版)は\verb|-n|オプションのみに対応しており、第一引数に``\verb|-e|''を与えれば表示する。また一方、AIXのechoコマンドはどちらにも対応していないため、第一引数に``\verb|-e|''や``\verb|-n|''を与えるとどちらも表示する。このように対応がバラバラだからだ。

\subsection*{エスケープシーケンスに反応する実装がある}

例えば\verb|\n|は改行を意味するエスケープシーケンスであるが、FreeBSDのechoはそのまま``\verb|\n|''と表示する。一方、Linuxのechoは\verb|-e|オプションが付けられた時のみ改行に置換される。また一方、AIXのechoは常に改行に置換する。

\textbf{AIXのechoはデフォルトでエスケープシーケンスを解釈する}のだ。「それPOSIX的にどうなの?」と困惑するかもしれないが、POSIXのechoのman\footnote{\verb|http://pubs.opengroup.org/onlinepubs/9699919799/utilities/echo.html|}にはちゃんとエスケープシーケンスの記述がある。

\subsection*{対策}

\textbf{どんな文字列が入っているかわからない変数を扱う場合}(ハイフンで始まらないとかエスケープシーケンスを含まないとわかっているならそのままでよい)、例えば次のようにprintfコマンドを使うなどして回避すること。

\paragraph{echoのオプション反応問題を回避する例(printfで代用)}  \\
\begin{frameboxit}{160.0mm}
\begin{verbatim}
	#! /bin/sh
	for arg in "$@"; do
	  printf '%s\n' "$arg"
	done
\end{verbatim}
\end{frameboxit}

\section{execコマンド}

注意すべきはexecコマンド経由で呼び出すコマンドに環境変数を渡したい時だ。

例えば、execコマンドを経由しない場合、コマンドの直前で環境変数を設定し、コマンドに渡すことができる。

\begin{screen}
	\verb|$ name=val awk 'BEGIN{print ENVIRON["name"];}'| \return \\
	\verb|val| \\
	\verb|$ |
\end{screen}

しかし、execコマンドを環境変数の直後に挿むと、何も表示されないシェルがある。

\begin{screen}
	\verb|$ name=val exec awk 'BEGIN{print ENVIRON["name"];}'| \return \\
	\verb|| \\
	\verb|$ |
\end{screen}

一部の環境のexecコマンドは、このようにして設定された環境変数を渡してくれないからだ。

もしexecコマンド越しに環境変数を渡したいのであれば、事前にexportで設定しておくこと。

\begin{screen}
	\verb|$ export name=val| \return \\
	\verb|$ exec awk 'BEGIN{print ENVIRON["name"];}'| \return \\
	\verb|val| \\
	\verb|$ |
\end{screen}

あるいは、execの後にenvコマンドを経由させるのでもよい。

\begin{screen}
	\verb|$ exec env name=val awk 'BEGIN{print ENVIRON["name"];}'| \return \\
	\verb|val| \\
	\verb|$ |
\end{screen}

\section{foldコマンド}
\label{allenvs:fold}

一般的に、ファイル名として``\verb|-|''を指定すると標準入力の意味と解釈されるが、本コマンドに対しては使わない方がよい。POSIXにはfoldコマンドでも、``\verb|-|''は標準入力だと解釈されると確かに書いてあるのだが、BSDの実装では真面目に``\verb|-|''というファイルを開こうとしてエラーになってしまう。

\section{grepコマンド}
\label{allenvs:grep}

\noindent
$\!\!\!\!\!$
\begin{grshfboxit}{160.0mm}
	俺は*BSDを使っているから、grepだってGNU拡張されていないBSD版のはず。ここで使えるメタ文字はどこでも使えるでしょ。
\end{grshfboxit}
と思っているアナタ。果たして本当にそうか確認してみてもらいたい。

\paragraph{アナタのgrepはホントにBSD版?}  \\
\begin{screen}
	\verb|$ grep --version| \return \\
	\verb|grep (GNU grep) 2.5.1-FreeBSD| \return \\
	\verb|| \\
	\verb|Copyright 1988, 1992-1999, 2000, 2001 Free Software Foundation, Inc.| \\
	\verb|This is free software; see the source for copying conditions. There is NO| \\
	\verb|warranty; not even for MERCHANTABILITY or FITNESS FOR A PARTICULAR PURPOSE.| \\
	\verb|| \\
	\verb|$ |
\end{screen}

なんと、GPLソフトウェア排除に力を入れているFreeBSDでも、grepコマンドはGNU版だ。関係者によれば、主に速さが理由で、grepだけは当面GNU版を提供するのだという。よって、POSIX標準だと思っていたメタ文字が実はGNU拡張だったということがある。代表的なものは``\verb|\+|''や``\verb|?|''や``\verb!\|!''である。

POSIX標準grepで使える正規表現メタ文字セットは、\verb|-E|オプション無しの場合にはBRE(基本正規表現)だけ。\verb|-E|オプション付きの場合にはERE(拡張正規表現)で規定されているものだけだ。詳しくは、本章第二部(正規表現)を参照のこと。

\section{headコマンド}

大抵の環境のheadコマンドは、\verb|-c|オプション(ファイルの先頭をバイト単位で切り出す)に対応している。しかし実は、\textbf{POSIXではheadコマンドに\verb|-c|オプションは規定されていない。}現に、正しく実装されていない環境も存在する\footnote{AIXでは最後に余計な改行コードが付く。}。

ちなみに、POSIXでもtailコマンドでは\verb|-c|オプションがきちんと規定されているので、headにだけ規定されていないのはちょっと不思議だ。

\subsection*{対策}

さて、それでは\verb|-c|オプションが使えない環境で何とかして同等のことができないものか……。大丈夫、ddコマンドでできる。

試に``12345''という5バイト(改行コードを加えれば6バイト)の文字列から先頭の3バイトを切り出してみよう。bs(ブロックサイズ)を3バイトとして、それを1つ(count)と指定すればよい。

\begin{screen}
	\verb!$ echo 12345 | dd bs=3 count=1 2>/dev/null! \return \\
	\verb|123$ |
\end{screen}

これは標準入力のデータを切り出す例だったが、\verb|if|キーワードを使えば実ファイルでもできる。

\begin{screen}
	\verb|echo 12345 > /tmp/hoge.txt| \return \\
	\verb|$ dd if=/tmp/hoge.txt bs=3 count=1 2>/dev/null| \return \\
	\verb|123$ |
\end{screen}

尚、ddコマンドは標準エラー出力に動作結果ログを吐くので、\verb|head -c|相当にするならddコマンドの最後に\verb|2>/dev/null|などと書いて、ログを捨てること。

\section{ifconfigコマンド}

これもPOSIXで規定されていないコマンドだし、最近のLinuxでは使われない傾向にあるコマンドであるが、全ての環境で動くことを目指すならまだまだ外せないコマンドである。

さて、実行中のホストに振られているIPアドレスを調べたい時にこのコマンドを使いたいことがあるが、各環境での互換性を確保するには2つのことに注意しなければならない。

\subsection*{パスが通っているとは限らない}

大抵の場合、ifconfigは\verb|/sbin|の中にある。しかし\textbf{多くのLinuxのディストリビューションでは一般ユーザーにsbin系のパスが通されていない。}だから、このコマンドを互換性を確保しつつ使いたい場合は、環境変数\verb|PATH|にsbin系ディレクトリー(\verb|/sbin|、\verb|/usr/sbin|)を追加しておく必要がある。

\subsection*{フォーマットがバラバラ}

ifconfigから返される書式が環境によってバラバラである。そこで、IPアドレスを取得するためのレシピを用意したので参照されたい。→\ref{recipe:ifconfig}(IPアドレスを調べる(IPv6も))参照

\section{killコマンド}
\label{allenvs:kill}

killコマンドで送信シグナルを指定する際は、名称でも番号でも指定できるわけだが、番号で指定する場合は気を付けなければならない。POSIXのkillコマンドのmanページ\footnote{\verb|http://pubs.opengroup.org/onlinepubs/9699919799/utilities/kill.html|}によれば、どの環境でも使える番号は\ref{tbl:signal_no}に記したもの以外保証されていない。


\begin{table}[htb]
  \caption{POSIXで番号が約束されているシグナル一覧}
  \begin{center}
  \begin{tabular}{ll}
    \HLINE
    Signal No. & Signal Name \\
    \hline
    0          & 0           \\
    1          & SIGHUP      \\
    2          & SIGINT      \\
    3          & SIGQUIT     \\
    6          & SIGABRT     \\
    9          & SIGKILL     \\
    14         & SIGALRM     \\
    15         & SIGTERM     \\
    \HLINE
  \end{tabular}
  \label{tbl:signal_no}
  \end{center}
\end{table}

「え、たったこれだけ!?」と思うだろうか。もちろんシグナルの種類がこれだけしかないわけではない。ただ、\textbf{その他のシグナルは名称と番号が環境によってまちまち}なのだ。例えば``SIGBUS''は、FreeBSDでは10だが、Linuxでは7、といった具合である。

従って、上記以外のシグナルを指定したい場合は名称("SIG"の接頭辞を略した文字列)で行うこと。使える名称自体は、POSIXのsignal.hに関する項\footnote{\verb|http://pubs.opengroup.org/onlinepubs/9699919799/basedefs/signal.h.html|}にも記されているとおり、豊富にある。

\subsection*{\verb|-l|オプションは避ける}

killコマンドで\verb|-l|オプションを指定すれば、使えるシグナルの種類の一覧が表示されるのはご存知のとおり。しかし、番号と名称の対応がこれで調べられるわけではない。Linuxだと丁寧に番号まで表示されるが、FreeBSDでは単に名称一覧しか表示されない(一応順番と番号は一致してはいるのだが)。

\section{mktempコマンド}
\label{allenvs:mktemp}

mktempコマンドもやはりPOSIXで規定されたものではない。よって、実際に使えない環境がある。

シェルスクリプトを本気で使いこなすにはテンポラリーファイルが欠かせず、そんな時に便利なコマンドがmktempなのだが……。どうすればいいだろうか。簡易的な対処と本格的な対処の2種類を用意した。

\subsection*{簡易的なmktemp}

\underline{一意性のみでセキュリティーは保証しない簡易的なもの}\footnote{もしセキュリティーを確保したい場合は良質な乱数源が必要となる。→\ref{allenvs:random_number}(乱数)参照}なら、下記のようなコードを追加しておけば作れる。

\paragraph{mktempコマンドが無い環境で、その「簡易版」を用意するコード}  \\
\begin{frameboxit}{160.0mm}
\begin{verbatim}
	type mktemp >/dev/null 2>&1 || {
	  mktemp_fileno=0
	  mktemp() {
	    (
	      filename="/tmp/${0##*/}.$$.$mktemp_fileno"
	      touch "$filename"
	      chmod 600 "$filename"
	      printf '%s\n' "$filename"
	    )
	    mktemp_fileno=$((mktemp_fileno+1))
	  }
	}
\end{verbatim}
\end{frameboxit}

簡単に解説しておこう。最初にmktempコマンドの有無を確認し、無ければコマンドと同じ使い方ができるシェル関数を定義するものだ。

ただし引数は無視され、必ず/tmpディレクトリーに生成されるので、それでは都合が悪い場合は適宜書き換えておくこと。それから、``\verb|mktemp_fileno|''という変数をグローバルで利用しているので書き換えないようにも注意すること。

\subsection*{本格的なmktemp}

POSIX版mktempコマンド\footnote{\verb|https://github.com/ShellShoccar-jpn/misc-tools/blob/master/mktemp|}を作ってしまったので、これをダウンロードして使えばよい。

書式はCoreutils版\footnote{\verb|https://www.gnu.org/software/coreutils/manual/html_node/mktemp-invocation.html#mktemp-invocation|}に似せてある。ただし動作パフォーマンス確保のため、/binや/usr/binの中に元々のmktempが存在すればそちらを使う(execする)ようにしてあるので、\textbf{あまり一般的でないオプションは使わない方がよい。}

\section{nlコマンド}

POSIXでも規定されている\verb|-w|オプションであるが、環境によって挙動が異なるので注意。
(尚、\verb|-w|オプションはPOSIXでデフォルト値が設定されているため、\textbf{このオプションを記述しなくても同様の問題が起こるので注意!}\footnote{POSIXの範囲ではないのだが、catコマンドの\verb|-n|オプションではこの問題は起こらないようだ。})

\verb|-w|オプションとは行番号に割り当てる桁数を指定するものであるが、問題は指定した桁数よりも桁があふれてしまった時である。
あふれた場合の動作はPOSIXでは定義されていないので、実装によって解釈が異ってしまったようだ。

2つの実装を例にとるが、まずBSD版のnlコマンドでは、溢れた分の上位桁は消されてしまう。

\paragraph{BSD版nlコマンドの場合}  \\
\begin{screen}
	\verb!$ yes | head -n 11 | nl -w 1! \return \\
	\verb|1       y| \\
	\verb|2       y| \\
	\verb|3       y| \\
	\verb|4       y| \\
	\verb|5       y| \\
	\verb|6       y| \\
	\verb|7       y| \\
	\verb|8       y| \\
	\verb|9       y| \\
	\verb|0       y| \\
	\verb|1       y| \\
	\verb|$ |
\end{screen}

一方、GNU版のnlコマンドでは、溢れたとしても消しはせず、全桁を表示する。

\paragraph{GNU版nlコマンドの場合}  \\
\begin{screen}
	\verb!$ yes | head -n 11 | nl -w 1! \return \\
	\verb|1       y| \\
	\verb|2       y| \\
	\verb|3       y| \\
	\verb|4       y| \\
	\verb|5       y| \\
	\verb|6       y| \\
	\verb|7       y| \\
	\verb|8       y| \\
	\verb|9       y| \\
	\verb|10      y| \\
	\verb|11      y| \\
	\verb|$ |
\end{screen}

行番号数字の直後につくのはデフォルトではタブ(``\verb|\t|'')なので、GNU版では桁数が増えるとやがてズレることになる。
BSD版はズレることはない代わりに上位桁が見えないので、何行目なのかが正確にはわからない。

\subsection*{対応方法}

AWKコマンドの組み込み変数である\verb|NR|を使うとよい。さらに、次のようにしてprintf関数を併用すれば、GNU版nlコマンドと同等の動作をする。

\paragraph{GNU版nlコマンドのデフォルトと同じ動作をする}  \\
\begin{frameboxit}{160.0mm}
\begin{verbatim}
	awk '{printf("%6d\t%s\n",NR,$0);}'
\end{verbatim}
\end{frameboxit}

\section{printfコマンド}
\label{allenvs:printf}

\subsection*{キャラクターコードによる即値指定}

互換性を重視するなら、\verb|\xHH|(``\verb|HH|''は任意の16進数)という16進数表記によるキャラクターコード指定をしてはいけない。これは一部のprintfの独自拡張だからだ。代わりに\verb|\OOO|(``\verb|OOO|''は任意の8進数)という3桁の8進数表記を用いること。

これは、AWKコマンドのprintf関数、sprintf関数についても同様である。

\subsection*{負の16進数}

負の値を16進数に変換すると環境によって結果が異なる。例えば-1を16進数に変換すると次のとおりだ。

\paragraph{32ビット実装の場合}  \\
\begin{screen}
	\verb!$ printf '%X\n' -1! \return \\
	\verb|FFFFFFFF|                   \\
	\verb|$ |
\end{screen}

\paragraph{64ビット実装の場合}  \\
\begin{screen}
	\verb!$ printf '%X\n' -1! \return \\
	\verb|FFFFFFFFFFFFFFFF|           \\
	\verb|$ |
\end{screen}

従って負の値を16進数に変換するのはあまり勧められないが、どうしてもしたいなら下8桁のみを取り出すべきだろう。もちろんその場合、-2147483648より小さい値は扱えない。

\section{psコマンド}

現在のpsコマンドは、オプションにハイフンを付けないBSDスタイルなど、いくつかの流派が混ざっているので厄介だ。

\subsection*{\verb|-x|オプションは避ける}

\verb|-x|オプションは「制御端末を持たないプロセスを含める」という働きを持つが、このオプションは使わない方がいい。そもそもPOSIXにおけるpsコマンドのmanページ\footnote{\verb|http://pubs.opengroup.org/onlinepubs/9699919799/utilities/ps.html|}には記載されていないし、少なくともGNU版とBSD版では解釈が異なるようだ。

例えばCGI(httpd)によって起動されたプロセス上で、何のオプションも付けずに自分のプロセスのみを表示した場合にあるGNU版で表示されたプロセスが、BSD版では\verb|-x|を付けた場合に初めて表示されるなどの違いがある。

結局のところ、互換性を重視するなら、大文字である\verb|-A|オプションを用いてとにかく全てを表示(\verb|-ax|に相当)させる方がよいだろう。

\subsection*{\verb|-l|オプションも避ける}

\verb|-l|オプションは、lsコマンドの同名オプションのように多くの情報を表示するためのものである。これはPOSIXのpsコマンドmanページにも記載されているし、実際主要な環境でサポートされているので問題なさそうだが、使うべきではない。理由は、表示される項目や順序がOSやディストリビューションによってバラバラだからだ。

\subsection*{\verb|-o|オプションほぼ必須}

\verb|-l|オプションを付けた場合の表示項目や順序がバラバラだと言ったが、実は\textbf{付けない場合もバラバラ}だ。どの環境でも期待できる表示内容といえば、

\begin{itemize}
  \item 1列目にPIDが来ること
  \item 行のどこかにコマンド名が含まれていること
\end{itemize}

\noindent
くらいなものだ。互換性を維持しながらそれ以上の情報を取得しようとするなら、\verb|-o|オプションを使って明確に表示させたい項目と順序を指定しなければならない。

\verb|-o|オプションで指定できる項目一覧についてはPOSIXのpsコマンドmanページ内の「STDOUTセクション」後半に記されている。(太小文字で列挙されている項目で、現在のところ"ruser"から"args"までが記されている)

\subsection*{補足.親プロセスID(PPID)}

Linuxでは、親プロセスIDが0になるのはPIDが1のinitだけだ。しかし、FreeBSD等ではinit以外のシステムプロセスの親プロセスIDも0になる場合がある。これは、psコマンドの違いというよりカーネルの違いであるが、互換性のあるプログラムを書くときには注意すべきところだ。

\section{readlinkコマンド}
\label{allenvs:readlink}

このコマンドは、与えられた引数がシンボリックリンクだった場合にその実体のパスを教えてくれるという便利なコマンドだ。しかし、残念ながらPOSIXのコマンド群には存在しない。

直接的な答えではないが、シンボリックリンクの実体を求めるシェルスクリプトのコードを\ref{recipe:fileoverwriting}(一時ファイルを作らずファイルを更新する)で紹介しているので参考にしてもらいたい。

\section{sedコマンド}

sedにもまたAWK同様に、複数の注意すべき点がある。

\subsection*{最終行が改行コードでないテキストの扱い}

試しに\verb@printf 'Hello,\nworld!'  | sed ''@というコードを実行してみてもらいたい。

\paragraph{BSD版sedの場合}  \\
\begin{screen}
	\verb@$ printf 'Hello,\nworld!'  | sed ''@ \return \\
	\verb|Hello,| \\
	\verb|world!| \\
	\verb|$ |
\end{screen}

\paragraph{GNU版sedの場合}  \\
\begin{screen}
	\verb@$ printf 'Hello,\nworld!'  | sed ''@ \return \\
	\verb|Hello,| \\
	\verb|world!$ |
\end{screen}

と、このように挙動が異なる。最終行が改行コードで終わっていない場合、BSD版は改行を自動的に挿入し、GNU版はしないようだ。

純粋なフィルターとして振る舞ってもらいたい場合にはGNU版の方が理想的ではあるが、すべての環境で動くことを目標にするならBSD版のような実装のsedとて無視するわけにはいかない。このようなsedをはじめ、AWKやgrep等、最終行に改行コードがなければ挿入されてしまうコマンドでの対処法を別のレシピとして記した。→\ref{recipe:nonLFterminated}(改行無し終端テキストを扱う)参照

\subsection*{使用可能なコマンド・メタ文字}

これも、GNU版は独自拡張されているので注意。

sedの中で使えるコマンドに関して迷ったら、POSIXのsedコマンドマニュアル\footnote{\verb|http://pubs.opengroup.org/onlinepubs/9699919799/utilities/sed.html|}を見る。また、sedが対応している正規表現メタ文字セットはBRE(基本正規表現)であり、本章第二部(正規表現)にその一覧を記してあるので参照してもらいたい。

\subsection*{標準入力指定の``\verb|-|''}

一般的に、ファイル名として``\verb|-|''を指定すると標準入力の意味と解釈されるが、本コマンドに対しては使わない方がよい。POSIXにはsedコマンドでも、``\verb|-|''は標準入力だと解釈されると確かに書いてあるのだが、BSDの実装では真面目に``\verb|-|''というファイルを開こうとしてエラーになってしまう。

\subsection*{ロケール}

\noindent
→\ref{allenvs:locale}(ロケール)を参照

\section{sortコマンド}

\noindent
→\ref{allenvs:locale}(ロケール)を参照

\section{tacコマンド・tailコマンド``-r''オプションによる逆順出力}

ファイルの行を最後の行から順番に(逆順に)並べたい時はtacコマンドを使うか、tailコマンドの\verb|-r|オプションのお世話になりたいところであろう。しかし、どちらも一部の環境でしか使えないし、もちろんPOSIXにも載っていない。

ではどうするか……。定番は、AWKで行番号を行頭に付けて、数値の降順ソートし、最後に行番号をとるという方法が無難だろう。

\paragraph{逆順出力するサンプルコード \#1}  \\
\begin{frameboxit}{160.0mm}
\begin{verbatim}
	#! /bin/sh

	# 逆順に並べたいテキストファイル
	cat <<TEXT > foo.txt
	a
	   b
	c
	TEXT

	cat foo.txt         |
	awk '{print NR,$0}' | # ←行頭に行番号をつける
	sort -k 1nr,1       | # ←行番号で降順にソート
	sed 's/^[0-9]* //'    # ←行番号を除去
\end{verbatim}
\end{frameboxit}

また、ソート対象のテキストデータが標準入力ではなくファイルであることがわかっているのであれば、exコマンドを使うという芸当もある。\footnote{bsdhack氏のブログ記事\verb|http://blog.bsdhack.org/index.cgi/Computer/20100513.html|より}

\paragraph{逆順出力するサンプルコード \#2}  \\
\begin{frameboxit}{160.0mm}
\begin{verbatim}
	#! /bin/sh

	# 逆順に並べたいテキストファイル
	cat <<TEXT > foo.txt
	a
	   b
	c
	TEXT

	ex -s foo.txt <<-EOF
	  g/^/mo0
	  %p
	EOF
\end{verbatim}
\end{frameboxit}

\section{test(``['')コマンド}
\label{allenvs:test}

どんな内容が与えられるかわからない文字列(シェル変数など)の内容を確認する時、最近のtestコマンドなら

\paragraph{シェル変数\$strの内容が``!''ならば``Bikkuri!''を表示}  \\
\begin{frameboxit}{160.0mm}
\begin{verbatim}
	[ "$str" = '!' ] && echo 'Bikkuri!'
\end{verbatim}
\end{frameboxit}

と書いても問題無いものが多い\footnote{さすがに\verb|$str|の中身が``\verb|(|''だった場合ダメなようだが。}。しかし、古来の環境では

\begin{screen}
	\verb|`[: =: unexpected operator`|
\end{screen}

というエラーメッセージが表示され、正しく動作しないものが多い。これは\verb|$str|に格納されている``\verb|!|''が、評価すべき文字列ではなく否定のための演算子と解釈され、そうすると後ろに左辺ナシの\verb|=|が現れたと見なされてエラーになるというわけだ。

testコマンドを用いて、全ての環境で安全に文字列の一致、不一致、大小を評価するには、文字列評価演算子の両辺にある文字列の先頭に無難な一文字を置く必要がある。

\paragraph{両辺にある文字列の先頭に無難な1文字を置けば、どこでも正しく動く}  \\
\begin{frameboxit}{160.0mm}
\begin{verbatim}
	[ "_$str" = '_!' ] && echo 'Bikkuri!'
\end{verbatim}
\end{frameboxit}

もっとも、単に文字列の一致、不一致を評価したいだけなら、testコマンドを使わずに下記のようにcase文を使う方がよい。上記のような配慮は必要ないし、外部コマンド(シェルが内部コマンドとして持ってる場合もあるが)のtestコマンドを呼び出さなくてよいので軽い。

\paragraph{case文で同等のことをする}  \\
\begin{frameboxit}{160.0mm}
\begin{verbatim}
	case "$str" in '!') echo 'Bikkuri!';; esac
\end{verbatim}
\end{frameboxit}

\section{trコマンド}
\label{allenvs:tr}

このコマンドは各UNIXの系譜基づく方言が強く残るコマンドの一種で、どこでも動くプログラムを無難に作るならなるべく使用を避けたいコマンドだ。

例えばアルファベットの全ての大文字を小文字に変換したい場合、
\begin{quote}
	\verb|tr '[A-Z]'  '[a-z]'| ← System V系での書式(運よくどこでも動く)\\
	\verb|tr 'A-Z'  'a-z'     | ← BSD系、POSIXでの書式
\end{quote}
という2つの書式がある。範囲指定の際にブラケット\verb|[|、\verb|]|が要るかどうかだ。
BSD系の場合、ブラケットは通常文字として解釈されるので、これを用いると置換対象文字として扱われてしまう。
しかしながらSystem V系の書式にあるブラケットは置換前も置換後も全く同一の文字なので幸いにしてどこでも動く。
従って、このようなケースでは前者の記述をとるべきだろう。

しかし、\verb|-d|オプションで文字を消したい場合はそうはいかない。
\begin{quote}
	\verb|tr -d '[a-z]'| ← System V系での書式(これはBSD系、POSIX準拠実装ではNG)\\
	\verb|tr -d 'a-z'  | ← BSD系、POSIXでの書式
\end{quote}

POSIXに準拠してないSystem V実装が悪いと言ってしまえばそれまでなのだが、
歴史の上ではPOSIXよりも早いので、それを言うのもまた理不尽というもの。ではどうすればいいか。

答えは、「sedで代用する」だ。上記のように、全ての小文字アルファベットを消したいという場合はこう書けばよい。
\begin{quote}
	\verb|sed 's/[a-z]//g'|
\end{quote}

しかしながら、改行コードで終わっていないテキストデータを与えると改行を付け足してしまうsed実装があるので、そういう可能性のあるデータを扱いたい場合は更に対策が必要だ。→\ref{recipe:nonLFterminated}(改行無し終端テキストを扱う)を参照

そこまでやるくらいだったら、範囲指定ではなく全部書いてしまえばいいと思うかもしれないが、もちろんそれでもいい。
\begin{quote}
	\verb|tr -d 'abcdefghijklmnopqrstuvwxyz'|
\end{quote}


\section{trapコマンド}
\label{allenvs:trap}

\noindent
→\ref{allenvs:kill}(killコマンド)参照


\section{whichコマンド}
\label{allenvs:which}

コマンドが存在すれば(パスが通っていれば)そのパスを返してくれるため、コマンドが無ければ無いなりにどの環境でも動くようなシェルスクリプトを書きた時などに重宝するコマンドだ。ところが、このwhichコマンドがPOSIX標準ではないというオチが待っている。

しかし諦めることはない。POSIXのコマンドで似た働きをするtypeというものがある。ただこれは人間向けの文章で出力されるので、それを解析してパスを取り出す必要がある。次のコードをシェルスクリプトの冒頭に追記しておけば、whichコマンドが存在しない場合のみ、typeコマンドに基づいたシェル関数版whichが登録される。

\paragraph{whichコマンドが無ければ同等品を追加するコード}  \\
\begin{frameboxit}{160.0mm}
\begin{verbatim}
	which which >/dev/null 2>&1 || {
	  which () {
	    (
	      ans=$(LANG=C LC_ALL=C type "$1" 2>/dev/null) || exit $?
	      case "$1"   in */*) printf '%s\n' "$1"        ; exit;; esac
	      case "$ans" in */*) printf '%s\n' "/${ans#*/}"; exit;; esac
	      printf '%s\n' "$1"
	    )
	  }
	}
\end{verbatim}
\end{frameboxit}


\section{xargsコマンド}
\label{allenvs:xargs}

\subsection*{改行なし終端データの扱い}

次の例を見てもらいたい。

\begin{screen}
	\verb!$ printf 'one two threee' | xargs echo! \return \\
	\verb|one two|                                        \\
	\verb|$ |
\end{screen}

単語が3つあるのだから、xargsはechoの後ろに``one''と``two''はもちろん、``three''も付けて実行してくれることを期待するが、最後の``three''が無視されてしまっている。じつはこのxargs実装、最後の単語の後にも改行やスペース等の列区切り文字を必要とするのである。

こういうxargs実装であっても確実に動作させるようにするには、例えばxargsの直前に\verb|grep ^|などを挿んでデータの終端に確実に改行が付くようにしてやることだ。

\begin{screen}
	\verb!$ printf 'one two threee' | grep ^ | xargs echo! \return \\
	\verb|one two three|                                           \\
	\verb|$ |
\end{screen}

\subsection*{空ループの有無}

標準入力から入ってきた文字列を引数にしてコマンドに渡すためのコマンドであるが、標準入力から空白以外が含まれた行が1行も渡ってこなかった場合、引数無しでコマンドを実行するxargs実装もあれば、コマンドを実行しないxargs実装もある。

\paragraph{BSD版の場合}  \\
\begin{screen}
	\verb!$ printf '  \n\n' | xargs echo 'foo'! \return \\
	\verb|$ |
\end{screen}

\paragraph{GNU版(多くのLinux)の場合}  \\
\begin{screen}
	\verb!$ printf '  \n\n' | xargs echo 'foo'! \return \\
	\verb|foo|                                          \\
	\verb|$ |
\end{screen}

xargsで呼び出される側のコマンドは引数0個で呼ばれるなど想定していない(Usageを表示したり戻り値0以外にしたりする)ものが多いので、前者の挙動の方が好ましいとは思うのだが、あるものはしょうがない。

一応、前者の動作に揃える\verb|-r|オプションというものがある(最近のFreeBSD版もこれを認識する)のだが、そんなオプションはPOSIXでは規定されていないがゆえ、それを付けて互換性を向上させようとすると逆に全ての環境で動く保証がなくなってしまうのが皮肉なところ。

\subsubsection*{対応方法}

さてどうするか……、これは対症療法しかない。すなわち、
\begin{enumerate}
  \item 引数0個で実行されてもエラー扱いしないようなコマンドにする。
  \item コマンドがエラー動作することを想定するような後続の処理にする。
  \item 標準入力に必ず有効かつ無害な行が入るようにする。
  \item 呼び出されるコマンドに無害な引数を付けておく。
\end{enumerate}
などを行う。

1番目の対処は、例えば呼び出すコマンドがrmなら\verb|-f|オプションを付けてエラー扱いを抑止するという方法だ。

\paragraph{対処方法1の例「rmコマンドをいちいちエラーで騒がせないようにする」}  \\
\begin{frameboxit}{160.0mm}
\begin{verbatim}
	find . -name '*.tmp' | xargs rm -f
\end{verbatim}
\end{frameboxit}

2番目の対処は、例えば戻り値が0以外でも即エラー扱いしないとか、標準エラーに流れてくるエラーメッセージやUsageを/dev/nullに捨てるというものだ。

\paragraph{対処方法2の例「rmコマンドがエラーで騒いでも無視する」}  \\
\begin{frameboxit}{160.0mm}
\begin{verbatim}
	(find . -name '*.tmp' | xargs rm) 2>/dev/null
\end{verbatim}
\end{frameboxit}

3番目の対処は、例えば呼び出すコマンドがgrepなどのファイルを読み込むだけのものであれば使える方法だ。例えば/dev/nullを読み出しファイルとして、標準入力の最初(最後に付加すると改行なし終端テキストだった場合に不具合が起こる)に付加すればよい。

\paragraph{対処方法3の例「grepに無害なファイル/dev/nullを読み込ませる」}  \\
\begin{frameboxit}{160.0mm}
\begin{verbatim}
	# grepの場合は後述の4番目の対処方法をお勧めする
	find . -name '*.txt' | awk 'BEGIN{print "/dev/null"} 1' | xargs grep '検索キーワード'
\end{verbatim}
\end{frameboxit}

4番目の対処は、手段が若干異なるだけで目的は3番目と同じだ。先程のgrepの例ならこう書き直す。短く書けるし、先程紹介した対処方法よりもお勧めだ。

\paragraph{対処方法4の例「grepに無害なファイル/dev/nullを読み込ませる(推奨)」}  \\
\begin{frameboxit}{160.0mm}
\begin{verbatim}
	find . -name '*.txt' | xargs grep '検索キーワード' /dev/null
\end{verbatim}
\end{frameboxit}

grepコマンドの場合は特にこちらを勧める。理由は、grepコマンドは、検索対象のファイルが1個だけ指定された場合と複数指定された場合で挙動を変えるからだ。具体的には、検索キーワードが見つかった時、1個だけだった場合はファイル名を表示しないのに対し、複数個だった場合には行頭にファイル名を併記する。

上記のように記述しておけば、grepコマンドは常に複数個指定されたと見なすので、findコマンドで見つかったファイルの数が1個であっても2個以上であっても、必ず行頭にファイル名を併記するようになり、動作が統一される。

\subsection*{引数文字列の扱い}

xargsに\verb|\\\'|という文字列を与えると、例えばFreeBSDのxargsとLinuxのxargsで異なった結果を返す。

\paragraph{FreeBSDの場合}  \\
\begin{screen}
	\verb!$ printf '\\\\\\'"' " | xargs printf! \return \\
	\verb|'|                                            \\
	\verb|$ |
\end{screen}

\paragraph{GNU版(多くのLinux)の場合}  \\
\begin{screen}
	\verb!$ printf '\\\\\\'"' " | xargs printf! \return \\
	\verb|\'|                                           \\
	\verb|$ |
\end{screen}

実はFreeBSDのxargsコマンドは、引数文字列を\verb|$@|(ダブルクォーテーションなし)のように渡してシェルのエスケープ処理を受けるのに対し、Linuxの(GNU版の)xargsコマンドは\verb|"$@"|(ダブルクォーテーションあり)のように渡すので、シェルのエスケープ処理を受けない。だから結果として、Linux上ではバックスラッシュが1個残るのだ。

ではどうするか。確実な方法は、エスケープ処理される文字を使わないことだ。バックスラッシュはいたしかたないとして、例えばシングルクォーテーションは\verb|\047|などと表現した文字列がprintfに渡るようにすればよい。ただしバックスラッシュも、引数としてシェルに解釈される時やprintfに解釈される時などにエスケープ処理を受けるので十分注意すること。

\section{zcatコマンド}
\label{allenvs:xargs}

zcatは、\verb!gunzip | cat!相当だと思っている人も多いかもしれないが違う!それはGNU拡張であり、\textbf{本来のzcatは\verb!uncompress | cat!相当}である。

従って、次のようにしてgzip圧縮されたデータを与えるとエラーを返すzcatコマンド実装がある。
\begin{screen}
	\verb!$ echo hoge | gzip | zcat! \return \\
	\verb|stdin: not in compressed format|   \\
	\verb|$ |
\end{screen}

全ての環境のzcatコマンドを想定するなら、compressコマンドで圧縮したデータを与えること。
\begin{screen}
	\verb!$ echo hoge | compress | zcat! \return \\
	\verb|hoge|                                  \\
	\verb|$ |
\end{screen}

\subsection*{ファイルを経由する場合の注意点}

compressコマンドは元データがファイルの場合、圧縮してもサイズが小さくならないと判明すると圧縮をしないという性質がある。そのため、普通に使うと次のような事故が起きる恐れがある。
\begin{screen}
	\verb!$ echo 1 >  hoge.txt! \return          \\
	\verb!$ compress hoge.txt! \return           \\
	\verb! -- file unchanged! ← サイズが小さくならないので圧縮ファイルは作られなかった \\
	\verb!$ zcat hoge.txt.Z! \return             \\
	\verb!hoge.txt.Z: No such file or directory! \\
	\verb!$ !
\end{screen}
これを防ぐためには、\verb|-f|オプションを付ければよい(\verb|compress -f|とする)。