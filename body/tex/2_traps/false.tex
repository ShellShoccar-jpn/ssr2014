\section{【緊急】falseコマンドの深刻な不具合}

\subsection*{問題}
\noindent
$\!\!\!\!\!$
\begin{grshfboxit}{160.0mm}
	falseコマンドに深刻なセキュリティーホールがあると聞いたが、一体どういう事か?
\end{grshfboxit}

\subsection*{回答}

2014年の今年、コンピューターセキュリティー史上一、二を争うであろうとても深刻なセキュリティーホールが見つかった。それがなんとfalseコマンドであった。後述の情報を読み、直ちに対応してもらいたい。


\subsection*{詳しい経緯}

2014年4月1日、``Single UNIX Specification''\footnote{\verb|http://www.unix.org/what_is_unix/single_unix_specification.html|}を策定しているThe Open Group\footnote{\verb|http://www.opengroup.org/|}が、falseコマンドに不具合を見つけたことを発表した。しかもそれは、コンピューターセキュリティー史上一、二を争うほどに深刻で、これまで一生懸命対策を講じてきた世界中のセキュリティー対策者達を一気に脱力させるほどのインパクトだという。

その理由の一つはまず、影響範囲があまりにも広いということ。なんと\textbf{falseコマンドを実装しているほぼ全てのUNIX系OS}がこの問題を抱えており、与える影響は計り知れないというのだ。

さらに深刻な理由は、この問題が発生したのは恐らくfalseコマンドの最初のバージョンで既にあったということである。最初のPOSIXには既にfalseコマンドが規定されており、それは1990年のことであるから、\textbf{どう短く見積もっても24年間この問題が存在していた}ことになる。

falseコマンドは、実行すると「偽」を意味する戻り値を返すだけというこれ以上無いほどに単純なコマンドであったために、まさかそこに脆弱性があろうとは長年誰も気付かなかったのである。

\subsubsection*{不具合の内容}

肝心の不具合の内容であるが、それは次のとおりだ。発表内容を引用する。\\

\noindent
\begin{frameboxit}{160.0mm}
現在のfalseコマンドのmanによれば、このコマンドは戻り値1(偽)を返すとされています。\\

しかしそれは、当然1を返してくるものと期待しているユーザーに対して\textbf{正直な動作}をしており、これはfalse(偽る)というコマンド名に対して実は不適切な動作をしています。
\end{frameboxit}

つまり、falseというコマンドの名に反してユーザーの命令に忠実に動いてしまっていたのだ。これは、「名は体を表す」が重要とされるコマンド名として、ましてやそれがPOSIXで規定されるコマンドとして、あってはならないことだ。

\subsubsection*{今後の対応方針}

今回の不具合報告に合わせ、The Open Groupのスポークスマンは次の声明を発表した。\\

\noindent
\begin{frameboxit}{160.0mm}
確かに前述のとおりの不具合は見つかったものの、falseというコマンドの長い歴史からすればとても「いまさら」な話である。また、あまりに普及率が高いコマンドで及ぼす影響も甚大であることからも、もし修正を実施したとなれば「いまさら仕様を変えるなよ」と世界中から白い目で見られることは必至である。\\

 我々もいまさらそんな苦労と顰蹙を買いたくないし、それに、今日(4/1)が終われば世の中もきっと我々の発表を無かったことにしてくれるに違いないと思うので、とりあえず今日をひっそり生きようと思う。
\end{frameboxit}

このようにThe Open Groupは\textbf{「それは断じて仕様である」メソッドの発動を示唆}しており、falseコマンドの動作は結局そのままになるものと見られている。従って、この問題に対しては各ユーザーが個別に対応し続けて行かねばならぬようだ。\\

\noindent
情報元\verb|;-p|→\verb|https://twitter.com/uspmag/status/450658253039366144|
