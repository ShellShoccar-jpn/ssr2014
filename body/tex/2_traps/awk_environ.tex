\section{標準入力以外からAWKに正しく文字列を渡す}

\subsection*{問題}
\noindent
$\!\!\!\!\!$
\begin{grshfboxit}{160.0mm}
	AWKに値を渡したいのだが、\verb|-v|オプションで渡しても、シェル変数を使ってソースコードに即値を埋め込んでも
	一部の文字が化けてしまう。どうすればよいか。ただし、標準入力は他のデータを渡すのに使っており、使えない。
	\begin{quote}
		\verb|str='\n means "newline"'          | ←渡したい文字列 \\
		\verb|| \\
		\verb|awk -v "s=$str" 'BEGIN{print s}'  | ←\verb|\n|がうまく渡せない \\
		\verb|| \\
		\verb|awk 'BEGIN{print "'"$str"'"}'     | ←\verb|\n|も\verb|"newline"|もうまく渡せない(エラーにもなる)
	\end{quote}
\end{grshfboxit}

\subsection*{回答}
環境変数として渡し、AWKの組込変数\verb|ENVIRON|で受け取る。問題文の``\verb|\n means "newline"|''を渡したいのであれば、こう書けばよい。\\
\begin{verbatim}
	str='\n means "newline"' awk 'BEGIN{print ENVIRON["str"]}'
\end{verbatim}
既にシェル変数に入っているのであればexportして環境変数化してから渡してもいいし、それが嫌ならコマンドの前に仮の環境変数(例えば``\verb|E|'')を置いて渡したり、envコマンドで渡してもいいだろう。\\
\begin{verbatim}
	str='\n means "newline"'

	export str
	awk 'BEGIN{print ENVIRON["str"]}'

	E=str awk 'BEGIN{print ENVIRON["E"]}'

	env E=str awk 'BEGIN{print ENVIRON["E"]}'
\end{verbatim}

\subsection*{解説}

何らかの事情でAWKに値を渡したいとなったら、手段はいくつかある。
\begin{enumerate}
  \item \verb|-v|オプションでAWK内の変数を定義して渡す
  \item AWKのコードに埋め込んで即値として渡す
  \item 標準入力から渡す
  \item 環境変数として渡す
\end{enumerate}
1番目は定番で、2番目も(筆者は)よくやる方法だ。だが、バックスラッシュを含む文字が化けるという問題がある。2番目に関しては問題文に示したように、ダブルクォーテーションを含む場合に単純な文字化けでは済まず、セキュリティーホールを生みかねない誤動作を招く。従ってこれらの方法はどんな文字が入っているかわからない文字列を渡すのには使えない。3番目の標準入力を使えれば安全なのだが、メインのデータを受け取るために既に使用中という場合もある。そうなると残る選択肢が4番目の環境変数というわけだ。

AWKは起動直後、環境変数を``ENVIRON''という名の組込変数に格納してくれる。これは連想配列なので環境変数名をキーにして読み出す。通常はこの変数によって現在設定されているロケールやコマンドパスを知るのに使うところだが、もちろんユーザーが自由に環境変数を定義してもよい。しかも都合の良いことに、全ての文字が一切エスケープずに伝わる。例えばこのようにして、シェルの組込変数IFS\footnote{文字列の列区切りと見なす文字列を定義しておく環境変数。for構文等で参照される。デフォルトでは半角スペースとタブ、そして改行コードが入っている。}を伝えることもできる。
\begin{screen}
	\verb!$ ifs="$IFS" awk 'BEGIN{print "(" ENVIRON["ifs"] ")",length(ENVIRON["ifs"]) }'! \return \\
	\verb!(          ! ← 括弧の中に空白、タブ、改行が表示され、 \\
	\verb!) 3        ! ← その直後に変数のサイズ(文字数)が3であると表示された。 \\
	\verb!$ !
\end{screen}

どんな文字が入っているかわからない文字列を安全に渡したい場合に知っておきたい手段だ。