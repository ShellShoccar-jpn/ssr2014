%\title{ }
%\author{ \\  \\  \\  \\  \\  \\  \\ 松浦リッチ研究所 \\ リッチ・ミカン}
%\maketitle


\thispagestyle{empty}
\begin{center}

 \\
 \\
 \\
 \\
 \\
 \\
 \\
 \\
 \\

\noindent
%%\Huge{\rellarge \textbf{Shell Script}\relsmall }
\Huge{\textbf{Shell Script ライトクックブック}} \\
\Huge{\textbf{ 2014}}
 \\
 \\
\noindent
\textbf{\LARGE{ }}

 \\
 \\
 \\
 \\
 \\
 \\

\noindent
\Large{リッチ・ミカン 著}

 \\
 \\

\begin{figure}[!h]
	\begin{center}
		\vspace{-1cm}
		\includegraphics*[scale=0.25]{tex/0_introduction/figs/MRL_logo_2007_outlined.eps}
		\vspace{-5cm}
	\end{center}
\end{figure}


\end{center}
\clearpage

\chapter{まえがき}

\section*{本書制作の経緯}

2005年、本書の第一弾となる「Shell Script ライトクックブック」という同人誌を発売した。当時、UNIXサーバー管理をしながら手の空いた時に書き貯めたレシピを収録したものだが、世にある厚い本とは対照的にレシピの数は少なかった。こんな薄い本がどれだけ需要あるだろうかと不安は感じていたものの不思議とよく売れた。しかもその後この本は商業誌となり、USP出版から「シンプルレシピ54」として今も発売されている。こちらを買ってくれた皆さんもありがとう。

そのUSP出版とはUSP研究所という会社の出版部門であるのだが、この会社が\textbf{エクストリームな変態企業(褒め言葉)}だったのである。シェルスクリプト以外の言語やデータベースミドルウェアを使わず、システム開発をやるというのだ。しかも話を聞けば、無印良品や東急ハンズ、ローソンなど、有名企業もバリバリ採用しているということで、どうやら遊びではないらしいことがわかった。

もともとこんな本を出すほどにシェルスクリプトが好きだったためUSP研究所とも付き合うようになり、シェルスクリプト総合誌``USP MAGAZINE''(現「シェルスクリプトマガジン」)を立ち上げた。これは2013年まで同人誌としコミケでも頒布していたが、それを買ってくれた皆さんもありがとう。そして「日本のインターネットの父」と称される村井純や、あの``K\&{}R''こと``The C Programming Language''の著者でもあるブライアン・カーニハンさえもビックリさせるほどの(実話)エクストリームなシェルスクリプトの世界を取材するとともに、自分自身も浸っていった。

そして2013年、その浸り具合は行くとこまで行ってしまった。私は\textbf{POSIX原理主義こそ至上である}という思想を唱えるようになってしまったのである。USP社の歴史を聞けば、昔は標準コマンドで足りない機能をsedやAWKなどを駆使して作っていたのだという。別にUSP社はUNIX哲学こそ唱えど、POSIX原理主義を唱えているわけではない。だが今ほどUNIX系OSの互換性が整備されていなかった時代に、業務の都合で様々なUNIX系OSを渡り歩かざるを得ない必要に迫られ、どの環境でも使えるような書き方をしていたのだという。その話に感銘を受けた私が勝手に唱えているだけなのだが、実際に頑張ってみると本当にいろいろなアプリケーションが作れてしまった。2013年は\textbf{POSIXの範囲でショッピングカートを作ることに成功}した。2014年の今年は、ただ一つPOSIX範囲外のcurlコマンドの力を借りるだけで、東京メトロのオープンデータ(JSON)を受信し、運行情報をブラウザーに表示するWebアプリケーションも作れてしまった。

こうしてソフトウェアを作っている間に貯まったレシピを書き記したのが本書である。POSIXをしゃぶり尽くす様を見て、存分に楽しんでもらいたい。

\section*{POSIX原理主義がすばらしい三つの理由}

私はPOSIX原理主義を唱えているわけだが、単なる自己満足で唱えているわけではない。POSIX原理主義を貫くと得られる様々な恩恵があるから唱えているのである。そのうちの最も大きな三つをここで述べよう。これらの理由は、サーバー管理で日々苦労している人ほど身に滲みる内容であるはずだ。

\subsection*{どこでも動く}

POSIXの範囲で書かれたソフトウェアはUNIXと名乗るOS上ならどこでも動く。なぜなら、「UNIXを名乗るOSなら、最低限この仕様は守りましょう」という仕様がPOSIXだからである。言いかえれば「UNIX系OSの最大公約数」だ。

「bashに脆弱性が見つかったので直ちに別のシェルに乗り換えたい」とか、「利用していたレンタルサーバー業者が急に倒産してしまったので別のサービスに乗り換えざるを得ないが、OSが替わってしまう」という外的要因に見舞われても痛くも痒くもない。

\subsection*{コンパイルせずに動く}

POSIXの範囲で書かれたソフトウェアを動かすのにコンパイルという作業は不要である。なぜなら、POSIXという要件を満たすために必要なコマンドは全てコンパイルされて、揃っており、揃っているからこそPOSIXを名乗れるからだ。

それゆえに、他のソフトでありがちな、「あっちのOSに持っていったらコンパイルが通らない!」といって悩むこともないし、コンパイル作業が要らないゆえにインストール作業はコピーだけということになり、あっという間に終わる。前述のような理由でホストの引っ越しを迫られたとしても重い腰を上げる必要などない。(そもそもデータベースを使っていないので、エクスポートやインポートといった作業も無い)

\subsection*{10年後も20年後も、たぶん動く}

POSIXの範囲で書かれたソフトウェアは、10年、20年の規模で長期間動き続ける。なぜなら、POSIXは、全てのUNIX系OSに準拠させるための最低限の仕様であるから、一つ二つのOSの都合で軽々と変更するわけにはいかないからだ。

OS独自のコマンドや他言語、データベースに依存したアプリケーションだと、利用しているそれらのソフトウェアのバージョンが0.1上がるだけで正常に動かなくなってしまったということが珍しくない。一方、POSIXの範囲だけで作られたアプリケーションであれば、バージョンアップの影響などまず受けない。そもそもバージョンアップ作業を強いられるのはOSくらいだ。


\section*{本書のポリシーおよび対象とする環境・ユーザー}

序章の最後に、本書が対象としている読者について記しておく。

\subsection*{全てのUNIX系OSで使えること --極力POSIXの範囲で済ませる--}

本書が対象とする環境は基本的にシェルスクリプトが動く全てのUNIX系OSである。シェルスクリプトとは、範囲の差こそあれどUNIX系OSであればどこでも、しかもインストール直後から動く言語であるのだから、その全て環境のユーザーに役立つ書籍でありたい。

そのため、基本的にはどの環境でも使える手法、つまりPOSIXの範囲のレシピ紹介する。より具体的には、``IEEE Std 1003.1''\footnote{``ieee''と``POSIX''という単語で検索すれば記載しているWebページに辿り着く。執筆時は2013年版が最新である。}で規定されているコマンドやオプション等にとどめる。PerlやPython等、多言語などもってのほかだ。そもそもPerlを許すなら、始めから全てPerlでやってしまった方が早いだろう。

そうは言うものの、もちろん全てのUNIX系OS上で本書で紹介しているレシピを試食できたわけではないため、お使いの環境よってはご賞味頂けないレシピがあるかもしれない。実のところこちらでの試食はFreeBSD 9~10、それにCentOS 5~7とAIXでしか行っていない。どうかご容赦頂きたい。

\subsection*{シェルスクリプトの基礎が身についている方であること}

本書はレシピ集である。つまり「やりたいことに対して、どのような機能を使い、また工夫をすればそれが実現できるか」を紹介する本である。従ってシェルスクリプトやコマンドにどんな機能が用意されているかを知っていなければ、本書のレシピを理解し、活用することは難しいと思う。できればその部分についてもページを割いて説明したいところではあるが、第一弾と同様に薄い本にするために省くこととした。

シェルスクリプトにまだあまり馴染みの無い方には不便を掛けてしまい大変申し訳ないが、本書のレシピを理解するのが難しいのであればWeb上のシェススクリプトについて解説しているページなどと一緒にご覧頂きたい。


\section*{``Open usp Tukubai''と、そのクローンについて}

本書を読み進めていくと``Open usp Tukubai''という用語が出てくる。これは、冒頭でも話題にしたUSP研究所からリリースされているシェルスクリプト開発者向けコマンドセットの名称である\footnote{公式サイト→\verb|https://uec.usp-lab.com/TUKUBAI/CGI/TUKUBAI.CGI?POMPA=ABOUT|。尚、恐ろしいほどの処理速度を発揮する有償版``usp Tukubai''というものも存在する。}。このコマンドセットは、シェルスクリプトをプログラミング言語として強化するうえで大変便利なものであり、本書で紹介するレシピのいくつかでは、そこに収録されているコマンド(Tukubaiコマンド)を利用している。

しかしながらオリジナル版のTukubaiコマンドは、全てその中身がPythonで書かれており、本書が提唱するPOSIX原理主義を貫くことができない。そんな中、やはりPOSIX原理主義に賛同している一人である321516さんから、Tukubaiコマンドの何割かをPOSIXで作り直したという話を教えてもらった(現在も移植中とのこと)\footnote{GitHub上で公開中→\verb|https://github.com/321516/Open-usp-Tukubai/tree/master/COMMANDS.SH|}。本書のレシピで利用しているTukubaiコマンドは全て、POSIXクローン版として移植の完了しているものであるので、安心してPOSIX原理主義の実力を見てもらいたい。


\section*{おことわり}

細心(最新)の注意を払ってはいるものの、間違った記憶、あるいは執筆後に仕様が変更されることによって\textbf{正しく動作しない内容が含まれている可能性}がある。不幸にもなおそのような箇所を見つけてしまった場合は下記の宛先へこっそりツッコミなどお寄せ頂きたい.

\begin{verbatim}
	richmikan@richlab.org
\end{verbatim}
