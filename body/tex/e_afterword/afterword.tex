\chapter*{あとがき}
\addcontentsline{toc}{chapter}{あとがき}

\subsubsection*{● カバーの説明}

\noindent
本書の表紙の動物はユリカモメです。カモメの一種ですが、くちばしと脚が赤いのが特徴です。古くは都鳥(みやこどり)とも呼ばれたようです。

\noindent
渡り鳥であり、日本には11月頃にやってきて4月頃まで越冬のため滞在します。このイラストは、東京の川の止まり木に一人(一鳥)佇む冬羽のユリカモメです。

\noindent
そして頭部の黒い夏羽に生え変わると、より高緯度の地域へ、繁殖のために旅立ちます。北米などでは、夏羽になった夏季に見られるため、ユリカモメの英語名は``black-headed gull''(直訳すればクロアタマカモメ)といいます。

\noindent
ところで、「ユリカモメ」と聞いて乗り物しか思い浮かばない人は、鉄分過多、あるいは有明病の疑いがありますのでご注意を。

\begin{figure}[!h]
	\begin{center}
		\vspace{1cm}
		\includegraphics*[scale=0.30]{tex/e_afterword/figs/yurikamome.eps}
		\vspace{-5cm}
	\end{center}
\end{figure}


\clearpage
\subsubsection*{● 著者コメント}
 \\
\noindent
\textbf{リッチ・ミカン} \\
\noindent
同人活動を始めたのが2002年で、気が付けばもう12年も続いている。10年目には名著``sed \& AWK''の著者でオ○イリー創業者の一人にインタビュー(ついでに小誌を見せる)という奇跡まで果たしたが、始めた頃にこんな未来が想像できただろうか?いいや、一発屋でせいぜい数年の同人作家生活だと思っていた。\\

\noindent
時代の移り変わりが特に激しいコンピューターの分野で同人作家が続いているからには、12年前には聞きもしなかった新技術を話題にした本を書いているかと思いきや!……今書いているのは1970年代に誕生し、1990年頃にPOSIXとして標準型がまとめられたUNIXである。しかもその「POSIX原理主義を受け入れよ」と言っている。一番最初に書いた本は1983年生まれのMSXパソコンの話であったから、時代が進むどころか遡っている。同人活動を始めた頃にこんな未来が想像できただろうか?いいや、新技術についていけずに35歳を待たずしてコンピューターエンジニアを引退するものと思っていた。\\

\noindent
今から12年後は何をやっているんだろうか?この調子ならひょっとすると10年後にティム・オ○イリーに会って、計算尺やそろばん、あるいはクルタ計算機の同人誌を見せびらかしているかもしれない。\\

\noindent
E-mail : \verb|richmikan@richlab.org| \\


\subsubsection*{● 表紙担当者コメント}
 \\
\noindent
\textbf{もじゃ} \\
\noindent
久しぶりの登場もじゃである。\\

\noindent
普段はもっぱらもじゃもじゃしている、もじゃSEである。実は密かに思い悩み、メンズ脱毛のカウンセリングに行ったところ、「成功する保証はない」と言われ、「ならばせめて成功率を教えてくれ」と語気を荒げて迫ったが、「わからない」とすげなく言われてしまい、ブチギレて帰ってきた次第である。理系としては成功率の統計もとっていないようなところは信用できないのである。\\

\noindent
それはそうと、誰にも同意してもらえないが、最近は人差し指と薬指の区別がつかなくて困っている。

\clearpage
\thispagestyle{empty}
\begin{figure}[!h]
	\vspace{153.0mm}
\end{figure}

\noindent
\begin{tabular}{ll}
	\multicolumn{2}{l}{
\textbf{
$\!\!\!\!$\Large{Shell Script ライトクックブック 2014} --- {\normalsize POSIX原理主義を貫く}
}
} \\
	\hline
	2014年12月30日							& 初版発行  \\
	 										&   \\
	\kintou{5zw}{著者}						& リッチ・ミカン \\
	\kintou{5zw}{表紙}						& もじゃ \\
	\kintou{5zw}{制作協力}					& 321516(三井浩一郎) \\
	\kintou{5zw}{印刷・製本}				& 株式会社イニュニック \\
	\kintou{5zw}{発行・発売}				& 松浦リッチ研究所 \\
	 										& \verb|http://richlab.org/| \\
	\kintou{5zw}{影の発行元}				& \textbf{秘密結社シェルショッカー日本支部} \\
	\hline
	\multicolumn{2}{l}{\footnotesize{Printed in Japan}} \\
	\multicolumn{2}{r}{\footnotesize{                                                            }} \\
\end{tabular}
