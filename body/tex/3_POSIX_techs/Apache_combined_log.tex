\section{Apacheのcombined形式ログを扱いやすくする}

\subsection*{問題}
\noindent
$\!\!\!\!\!$
\begin{grshfboxit}{160.0mm}
	Apacheのログファイル(combined形式)がある。
	しかしこれ、単純なスペース区切りのファイルではなく、
	大括弧(\verb|[~]|)やダブルクォーテーションで囲まれている区間は、1つのフィールドとされている。
	それゆえ、アクセス日時の列、User-Agentの列など、任意のフィールドを抽出することがとても面倒だ。
	簡単に取り出せるようにならないものか。
\end{grshfboxit}

\subsection*{回答}
sedコマンドを4回、trコマンドを2回通せばできる。通すと、各フィールド内の空白文字がアンダースコアに置換され、フィールド区切りとしての空白だけが残るので、以後AWKなどで簡単に列を抽出することができるようになる。

\paragraph{Apacheログを正規化するシェルスクリプト apacomb\_{}norm.sh}  \\
\begin{frameboxit}{160.0mm}
\begin{verbatim}
	#! /bin/sh

	# --- その前に、ちょっと下ごしらえ ---
	RS=$(printf '\036')             # 元々の改行位置を退避するための記号定義
	LF=$(printf '\\\n_');LF=${LF%_} # sedで改行コードを挿れるための定義
	c='_'                           # ここに空白の代替文字(今はアンダースコアにしている)

	# --- 本番 ---
	cat "$1"                                   | # 第一引数でApacheログを指定しておく
	sed 's/^\(.*\)$/\1'"$RS"'/'                |
	sed 's/"\([^"]*\)"/'"$LF"'"\1"'"$LF"'/g'   |
	sed 's/\[\([^]]*\)\]/'"$LF"'[\1]'"$LF"'/g' |
	sed '/^["[]/s/[[:blank:]]/'"$c"'/g'        |
	tr -d '\n'                                 |
	tr "$RS" '\n'
\end{verbatim}
\end{frameboxit}

このシェルスクリプトを通した後、日時フィールドが欲しければ\verb|awk '{print $4}'|、同様にHTTPリクエストパラメーターフィールドなら\verb|awk '{print $5}'|、User-Agentフィールドなら\verb|awk '{print $9}'|をパイプ越しに書き足せばよいわけだ。

\subsection*{解説}

ご承知のとおり、Apacheで一般的に使われているcombinedという形式のログはこんな内容になっている。\\
\begin{frameboxit}{160.0mm}
\begin{verbatim}
	192.168.0.1 - - [17/Apr/2014:11:22:33 +0900] "GET /index.html HTTP/1.1" 200 43206 "https://www.g
	oogle.co.jp/" "Mozilla/5.0 (Windows NT 6.1; WOW64) AppleWebKit/537.36 (KHTML, like Gecko) Chrome
	/34.0.1847.116 Safari/537.36"
\end{verbatim}
\end{frameboxit}

User-Agentフィールド(\verb|"mozilla/5.0 (Windows NT 6.1 ……  Safari/537.36"|の部分)が欲しいと思って、AWKで抽出しようとしても
\begin{screen}
	\verb|$ awk '{print $12}' httpd-access.log| \return \\
	\verb|"Mozilla/5.0| \\
	\verb|$ |
\end{screen}
となってしまって、全然使い物にならない。

しかし、そこは我らがUNIX。シェルスクリプトとパイプと標準コマンドであるsedとtrさえあればお手のものだ。他言語に走る必要など全く無い。

「回答」で示したシェルスクリプトに掛けてみるとご覧のとおりだ。
\begin{screen}
	\verb!$ cat httpd-access.log | apacomb_norm.sh! \return \\
	\verb|192.168.0.1 -  -  [17/Apr/2014:11:22:33_+0900] "GET_/index.html_HTTP/1.1" 200 43206 "h| \\
	\verb|ttps://www.google.co.jp/" "Mozilla/5.0_(Windows_NT_6.1;_WOW64)_AppleWebKit/537.36_(K| \\
	\verb|HTML,_like_Gecko)_Chrome/34.0.1847.116_Safari/537.36"| \\
	\verb|$ |
\end{screen}

\subsubsection*{各々のsedとtrは何をやってるのか?}

1つ1つ説明していこう。

\paragraph{sed \#1}
(加工の都合により、途中で一時的に改行を挿むので)元の改行を別の文字$<$0x1E$>$で退避させておく。

\paragraph{sed \#2}
ダブルクォーテーションで囲まれている区間\verb|"~"|があったら、その前後に改行を挿み、その区間を単独の行にする。

\paragraph{sed \#3}
ブラケットで囲まれている区間\verb|[~]|も同様に、前後に改行を挿んで、この区間を単独の行にする。

\paragraph{sed \#4}
ダブルクォーテーション、またはブラケットで始まる行は、先程行を独立させた区間なので、これらの行にある空白をそうでない文字列に置換する。

\paragraph{tr \#1}
改行を全部取り除く。

\paragraph{tr \#2}
退避させていた元々の改行を復活させる。

\subsubsection*{全部sedでやることもできる}

ちなみにtrコマンドをsedに置き換え、全てをsedにすることもできる。\\
\begin{frameboxit}{160.0mm}
\begin{verbatim}
	cat "$1"                                   |
	sed 's/^\(.*\)$/\1'"$RS"'/'                |
	sed 's/"\([^"]*\)"/'"$LF"'"\1"'"$LF"'/g'   |
	sed 's/\[\([^]]*\)\]/'"$LF"'[\1]'"$LF"'/g' |
	sed '/^["[]/s/[[:blank:]]/'"$c"'/g'        |
	sed 'N;$s/\n//g'                           |
	sed 's/'"$RS"'/'"$LF"'/g'
\end{verbatim}
\end{frameboxit}
trコマンドの方が速いので意味のあることではないが……。

\subsection*{コマンド化したものをGitHubにて提供中}

いちいち本書を読んで書き写すのも面倒であろうし、少し改良したものをGitHub上に公開した。よければ使ってみてもらいたい。

\begin{quote}
	\verb|https://gist.github.com/richmikan/7254345|
\end{quote}

スペースの代替文字が\verb|_|では気に入らない人向けに、オプションで指定できるよにしてある本格派だ。Apacheサーバー管理者は、これで少し幸せになれるかもしれない。

\subsection*{参照}

\noindent
→\ref{recipe:sed_LF}(sedによる改行文字への置換を、綺麗に書く)
