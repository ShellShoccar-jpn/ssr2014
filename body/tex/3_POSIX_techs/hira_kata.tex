\section{ひらがな・カタカナの相互変換}
\label{recipe:hira_kata}

\subsection*{問題}
\noindent
$\!\!\!\!\!$
\begin{grshfboxit}{160.0mm}
	名簿入力フォームで名前とふりがなが集まったのだが、
	ふりがなが人によってひらがなだったりカタカナだったりするので統一したい。
	どうすればいいか。
\end{grshfboxit}

\subsection*{回答}
全角・半角の相互変換と方針は似ていて、基本方針は
\begin{itemize}
  \item テキストデータを1バイトずつ読み、各文字が何バイト使っているのかを認識しながら読み進めていく。
  \item その際、対になるひらがなあるいはカタカナに変換可能な文字に遭遇した場合は置換する。
\end{itemize}
である。尚、ひらがなは全角文字にしか存在しない\footnote{MSXなど半角ひらがなを持っているコンピューターはあるのだけど一般的ではない。}ため、全角文字前提での話とする。半角カタカナを扱いたい場合は、\ref{recipe:han_zen}(全角・半角文字の相互変換)によって全角に直してから参照すること。

\subsubsection*{ひらがな→カタカナ変換コマンド``hira2kata''}

例によってPOSIXの範囲で実装し、コマンド化したものがGitHubに公開されている。``hira2kata''という名のコマンドだ。これは321516氏が、USP研究所のOpen usp Tukubaiというシェルスクリプト開発者向けコマンドセット\footnote{\verb|https://uec.usp-lab.com/TUKUBAI/CGI/TUKUBAI.CGI?POMPA=ABOUT|}にあるhan、zenコマンドのインターフェースに似せる形で作ったものである。もちろんPOSIXの範囲でだ。これをダウンロード\footnote{\verb|https://github.com/ShellShoccar-jpn/misc-tools/blob/master/hira2kata|にアクセスし、そこにあるソースコードをコピー\&{}ペーストしてもよいし、あるいは``RAW''と書かれているリンク先を「名前を付けて保存」してもよい。}して用いる。

例えば、次のようなテキストファイル(furigana.txt)があったとする。\\
\begin{frameboxit}{160.0mm}
\begin{verbatim}
	#No. フリガナ
	い   もがみ
	ろ   カガ
	は   ふぶき
	に   ムツ
	ほ   ぜかまし
\end{verbatim}
\end{frameboxit}

問題文にもあったように、回答者によってふりがなをひらがなで入力したりカタカナで入力したりまちまちになっている。回答者名で検索したいとなった時、検索する側はいちいちひらがなかカタカナかを区別したくない。このよう時に、次のようにしてhira2kataを使うのである。
\begin{screen}
	\verb!$ ./hira2kata 2 furigana.txt! \return \\
	\verb|#No. フリガナ| \\
	\verb|い   モガミ| \\
	\verb|ろ   カガ| \\
	\verb|は   フブキ| \\
	\verb|に   ムツ| \\
	\verb|ほ   ゼカマシ| \\
	\verb|$ |
\end{screen}

こうしておけば、全角カタカナで簡単に回答者名のgrep検索が可能になるし、50音順ソートもできるようになる。注意すべきは、この例ではhira2kataコマンドの第1引数に``\verb|2|''と書いてあるところである。これは、第2フィールドだけ変換せよという意味である。よって第1フィールドの数字はそのままになっている。仮に``\verb|2|''という引数無しにファイル名だけ指定すると、フィールドという概念無しに、テキスト中にある全てのひらがなを変換しようとする。よって、その場合第1フィールドの「いろは……」もカタカナになる。

尚、この\textbf{hira2kataコマンドはUTF-8のテキストにしか対応しておらず}、JISやShitf JIS、EUC-JPテキストには対応していない。そのような文字を扱いたい場合は、iconvコマンドを用いたり、nkfコマンド(POSIXではないが)を用いて予めUTF-8に変換しておくこと。

それから、このサンプルデータで「ほ」行の島風さんだけ右から書いてあるが、\textbf{こういうのはどーしよーもない!}昭和じゃないんだから左から書くように言っておいってもらいたい。

\subsubsection*{カタカナ→ひらがな変換コマンド``kata2hira''}

上の例ではカタカナに統一したが、逆にひらがなに統一してもよい。その場合は``kata2hira''という名のコマンドを使う。これも、321516氏がUSP研究所のOpen usp Tukubaiというシェルスクリプト開発者向けコマンドセット\footnote{\verb|https://uec.usp-lab.com/TUKUBAI/CGI/TUKUBAI.CGI?POMPA=ABOUT|}にあるhan、zenコマンドのインターフェースに似せる形で、POSIX範囲内で作ったものである。あわせてダウンロード\footnote{\verb|https://github.com/ShellShoccar-jpn/misc-tools/blob/master/kata2hira|にアクセスし、そこにあるソースコードをコピー\&{}ペーストしてもよいし、あるいは``RAW''と書かれているリンク先を「名前を付けて保存」してもよい。}しておくとよいだろう。

\subsection*{解説}

前のレシピで全角文字を半角文字に変換するコマンドが作れたのだから、半角文字に変換する代わりにひらがな・カタカナの変換をするのも大したことはない。

マルチバイト文字なので、半角・全角変換と同様に、1バイトずつ読んだ場合、それがマルチバイト文字の終端なのか、それとも途中なのかということを常に判断しなければならないのだが、その後の置換作業で一工夫してある。

半角全角変換の際は完全に連想配列に依存していたが、ひらがな・カタカナ変換においては、高速にするために使用を控えている。その代わりにキャラクターコードを数百番ずつシフトするような計算を行っている。UTF-8においては、ひらがなとカタカナはユニコード番号が数十バイト離れたところにそれぞれマッピングされている\footnote{『オレンジ工房』さんのUTF-8の文字コード表 全角ひらがな・カタカナというページ参照 \\ →\verb|http://orange-factory.com/sample/utf8/code3-e3.html|}ので、それを見ながらユニコード番号を数百番ずらして目的の文字を作っているというわけだ。

\subsection*{参照}

\noindent
→\ref{recipe:han_zen}(全角・半角の相互変換)
