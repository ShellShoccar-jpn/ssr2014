\section{ヒストリーを残さずログアウト}

\subsection*{問題}
\noindent
$\!\!\!\!\!$
\begin{grshfboxit}{160.0mm}
	今、\verb|rm -rf ~/public_html/*|というコマンドで公開webディレクトリーの中身をごっそり消した。 \\
	こんなおっかないコマンドはヒストリーに残したくないので、今回だけはヒストリーを残さずにログアウトしたい。
\end{grshfboxit}

\subsection*{おことわり}

このTipsは不作法だとして異論が出るかもしれない。私個人は、何か致命的なことが起こるとは思わないものの、
\textbf{ここで紹介するコマンドを打って何か不具合が起こったとしても苦情は受け付けない}ので予めご了承いただきたい。

\subsection*{回答}

ログインしたいと思った時、次のコマンドを打てばよい。

\begin{screen}
	\verb|$ kill -9 $$| \return
\end{screen}

\subsection*{解説}

``\verb|kill -9 <|\textit{プロセスID}\verb|>|''とは指定プロセスを強制終了するためのコマンド書式だ。
変数\verb|$$|は、今ログインしているシェルのプロセスIDを持っている特殊な変数であるため、これを強制終了することを意味する。

強制終了とは、対象プロセスに終了の準備をさせる余地を与えず瞬殺することを意味するから、
シェルに対してそれを行えば、ヒストリーファイルを更新する余地を与えずログアウトできるというわけだ。

簡単でしょ。
