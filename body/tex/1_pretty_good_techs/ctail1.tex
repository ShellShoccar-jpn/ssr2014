\section{テキストデータの最後の行を消す}

\subsection*{問題}
\noindent
$\!\!\!\!\!$
\begin{grshfboxit}{160.0mm}
	あるメールシステムから取得したテキストデータがあって、その最終行には必ずピリオドがある。邪魔なので取り除きたい。 \\
	だが、「行数を数えて最後の行だけ出力しない」というのも大げさだ。簡単にできないものか。
\end{grshfboxit}

\subsection*{回答}
「最後の1行」と決まっているなら、行数を数えなくともsedコマンド1個で非常に簡単にできる。
下記は、メール(に見立てたテキスト)の最終行をsedコマンドで取り除く例である。
\begin{screen}
	\verb!$ cat <<MAIL | sed '$d'! \return \\
	\verb|やぁ皆さん、私の研究室へようこそ。    | ↑                    \\
	\verb|以上                              | この3行が元のテキスト \\
	\verb|.                                 | ↓                    \\
	\verb|MAIL| \\
	\verb|やぁ皆さん、私の研究室へようこそ。| \\
	\verb|以上                              | ←2行目で終わっている \\
	\verb|$ |
\end{screen}

\subsection*{解説}

例えば標準入力から送られてくるテキストデータの場合、
普通に考えれば、一度テンポラリーファイルに書き落として行数を数えなければならないところだ。
あるいは「一行先読みして……。読み込みに成功したら先読みしていた行を出力して……」とやらなければならない。
どちらにしても、これを自分で実装するとなったら面倒臭い。

しかしsedコマンドは、始めからその先読みを内部的にやってくれている。
だから最終行に何らかの加工を施す``\verb|$|''という指示が可能である。
今は最終行を出力したくないのだから、sedの中で削除を意味する``d''コマンドを使う。
つまりsedで``\verb|$d|''と記述すれば最終行が消えるのである。

これを応用すれば、次のようにして最後の2行を消すことも可能だ。

\begin{screen}
	\verb!$ cat <<MAIL | sed '$d'| sed '$d'! \return \\
	\verb|やぁ皆さん、私の研究室へようこそ。| \\
	\verb|以上| \\
	\verb|.| \\
	\verb|MAIL| \\
	\verb|やぁ皆さん、私の研究室へようこそ。| \\
	\verb|$ |
\end{screen}

同様にして3行でも4行でも……。まぁ、だんだんとカッコ悪いコードになっていくが。