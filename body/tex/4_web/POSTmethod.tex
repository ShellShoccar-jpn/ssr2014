\section{CGI変数の取得(POSTメソッド編)}
\label{recipe:POSTmethod}

\subsection*{問題}
\noindent
$\!\!\!\!\!$
\begin{grshfboxit}{160.0mm}
	Webブラウザーから送られてくるCGI変数を読み取りたい。\\
	ただしPOSTメソッド(環境変数``REQUEST\_{}METHOD''が``POST''の場合)である。
\end{grshfboxit}

\subsection*{回答}
基本的には\ref{recipe:GETmethod}(CGI変数の取得(GETメソッド編))と同じである。
よってまず2つのコマンド(``cgi-name''及び``namread'')をダウンロード\footnote{\verb|https://github.com/321516/Open-usp-Tukubai/blob/master/COMMANDS.SH/cgi-name|、及び \\ \verb|https://github.com/321516/Open-usp-Tukubai/blob/master/COMMANDS.SH/nameread|にアクセスし、そこにあるソースコードをコピー\&{}ペーストしてもよいし、あるいは``RAW''と書かれているリンク先を「名前を付けて保存」してもよい。}する。

ここでも例をあげて説明しよう。今、Webブラウザーが次のようなHTMLに基づいてCGI変数を送ってくるものとする。
\paragraph{Webブラウザーが送信する元になるHTML}  \\
\begin{frameboxit}{160.0mm}
\begin{verbatim}
	<form action="form.cgi" method="POST">
	  <dl>
	    <dt>名前</dt>
	      <dd><input type="text" name="fullname" value="" /></dd>
	    <dt>メールアドレス</dt>
	      <dd><input type="text" name="email" value="" /></dd>
	  </dl>
	  <input type="submit" name="post" value="送信" />
	</form>
\end{verbatim}
\end{frameboxit}

これを取得するためのシェルスクリプトは次のようになる。
\paragraph{前述のフォームから送られてきたCGI変数を受け取るシェルスクリプト(form.cgi)}  \\
\begin{frameboxit}{160.0mm}
\begin{verbatim}
	#! /bin/sh
	
	Tmp=/tmp/${0##*/}.$$  # 一時ファイルの元となる名称
	
	dd bs=${CONTENT_LENGTH:-0} count=1 |
	cgi-name                           > $Tmp-cgivars  # 正規化し、一時ファイルに格納
	
	fullname=$(nameread fullname $tmp-cgivars)         # CGI変数"fullname"を取り出す
	email=$(nameread email $tmp-cgivars)               # CGI変数"email"を取り出す
	
	(ここで何らかの処理)
	
	rm -f $Tmp-*                                       # 用が済んだら一時ファイルを削除
\end{verbatim}
\end{frameboxit}

GETメソッドとの唯一の違いは、読み出す元が環境変数ではなく標準入力に代わったことだ。
プログラム上ではそれに対応するため、ddコマンドを用いるようになった点のみが異なっている。

\subsection*{解説}

基本的な解説は\ref{recipe:GETmethod}(CGI変数の取得(GETメソッド編))で済ませてあるので、同じことに関しては省略する。
ここではPOSTメソッドで異なる点についてのみ述べる。

先程も述べたように、POSTはCGI変数の格納されている場所が環境変数ではなく標準入力であるという点が唯一異なる。
標準入力から受け取るならcatコマンドでもいいような気がするが、安全を期してddコマンドを使うべきだ。

理由は、環境によっては運が悪いと標準入力からのデータを受け取るのに失敗してcatコマンド実行で止まってしまう恐れがあるからだ。
CGI変数文字列のサイズ(環境変数CONTENT\_{}LENGTH)が0である場合は読み取る必要がないのだが、
環境によっては0なのに読もうとすると止まってしまうことがあるようだ。そのためにこのようなやり方を推奨している。

\subsection*{参照}

\noindent
→\ref{recipe:GETmethod}(CGI変数の取得(GETメソッド編))\\
→\ref{recipe:file_upload}(Webブラウザーからのファイルアップロード)
