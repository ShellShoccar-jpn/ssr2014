\section{URLデコードする}
\label{recipe:URL_decode}

\subsection*{問題}
\noindent
$\!\!\!\!\!$
\begin{grshfboxit}{160.0mm}
	Webサーバーのログを見ていると、検索ページからジャンプしてきている形跡があった。しかし、検索キーワードはURLエンコードされた状態であり、デコードしないとわからないのでデコードしたい。
\end{grshfboxit}

\subsection*{回答}
そんなに難しい仕事でないから素直に書いて作る。基本的には正規表現で``\verb|%[0-9A-Fa-f]{2}|''を検索し、
見つかるたびにprintf関数を使ってその16進数に対応するキャラクターに置き換えればよい。AWKで書くならこんな感じだ。

\paragraph{URLデコードするコード}  \\
\begin{frameboxit}{160.0mm}
\begin{verbatim}
	env i- awk '
	BEGIN {
	  # --- prepare
	  LF  = sprintf("\n");
	  OFS = "";
	  ORS = "";
	  # --- prepare decoding
	  for (i=0; i<256; i++) {
	    l  = sprintf("%c",i);
	    k1 = sprintf("%02x",i);
	    k2 = substr(k1,1,1) toupper(substr(k1,2,1));
	    k3 = toupper(substr(k1,1,1)) substr(k1,2,1);
	    k4 = toupper(k1);
	    p2c[k1]=l;p2c[k2]=l;p2c[k3]=l;p2c[k4]=l;
	  }
	  # --- decode
	  while (getline line) {
	    gsub(/\+/, " ", line);
	    while (length(line)) {
	      if (match(line,/%[0-9A-Fa-f][0-9A-Fa-f]/)) {
	        print substr(line,1,RSTART-1), p2c[substr(line,RSTART+1,2)];
	        line = substr(line,RSTART+RLENGTH);
	      } else {
	        print line;
	        break;
	      }
	    }
	    print LF;
	  }
	}'
\end{verbatim}
\end{frameboxit}

1文字ではなく1バイトずつ処理する必要があるので``\verb|env -i|''をAWKの手前に付けて、ロケール環境変数の影響を受けないようにする。

\subsection*{解説}
文字を1バイト毎に、16進数2桁表現でアスキーコード化し、その先頭に``\verb|%|''文字を付けるエンコード方式を\textbf{パーセントエンコーディング}と呼ぶ。ただしURLに用いる文字のうち特殊な意味を持つものだけをパーセントエンコーディングするとともに半角スペースは``\verb|%20|''ではなく``\verb|+|''にエンコードする場合を、「URLエンコーディング」とか「URLエンコード」などと呼んだりする。これは、RFC 3986のSection2.1で定義されている。

このエンコーディングのルールさえ理解できれば、デコーダーを作ることなど大したことではない。
例えばWeb検索するとurlendecというパッケージ\footnote{\verb|http://www.whizkidtech.redprince.net/urlendec/|}が見つかる。
しかしPOSIX原理主義者たるものそういったものに安易に頼ってはいけない。
なんとこのツールは2014/12現在32bit専用らしい。
もしこのソフトを愛用している人が32bit環境から64bit環境に移行しようとしたら痛い目を見る。(かつての筆者)

\subsection*{参照}

\noindent
→RFC 3986文書\footnote{http://www.ietf.org/rfc/rfc3986.txt} \\
→\ref{recipe:URL_encode}(URLエンコードする)
