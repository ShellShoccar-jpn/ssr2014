\section{シェルスクリプトおばさんの手づくりCookie(読み取り編)}
\label{recipe:read_cookie}

\subsection*{問題}
\noindent
$\!\!\!\!\!$
\begin{grshfboxit}{160.0mm}
	クライアント(Webブラウザー)が送ってきたCookie情報を、シェルスクリプトで書いたCGIスクリプトで読み取りたい。
\end{grshfboxit}

\subsection*{回答}
Cookie文字列はCGI変数とよく似たフォーマットで、しかも環境変数で渡ってくるので\ref{recipe:GETmethod}(CGI変数の取得(GETメソッド編))がほぼ流用できる。しかしながら若干の相違点があるので、具体例を示しながら説明する。

例として掲示板のWebアプリケーションの場合を考えてみる。投稿者名とメールアドレスをそれぞれ``name''、``email''という名前でCookieに保存していたとすると、それを取り出すには次のように書けばよい。
\paragraph{掲示板の投稿者名とe-mailをCookieから取り出す}  \\
\begin{frameboxit}{160.0mm}
\begin{verbatim}
	#! /bin/sh
	
	Tmp=/tmp/${0##*/}.$$                                # 一時ファイルの元となる名称
	
	printf '%s' "${HTTP_COOKIE:-}" |
	sed 's/&/%26/g'                |
	sed 's/[;, ]\{1,\}/\&/g'       |
	sed 's/^&//; s/&$//'           |
	cgi-name                       > $Tmp-cookievars    # 正規化し、一時ファイルに格納
	
	name=$(nameread name $tmp-cookievars)               # Cookie変数"name"を取り出す
	email=$(nameread email $tmp-cookievars)             # Cookie変数"email"を取り出す
	
	(ここで何らかの処理)
	
	rm -f $Tmp-*                                        # 用が済んだら一時ファイルを削除
\end{verbatim}
\end{frameboxit}

GETメソッドとの違いは、
\begin{itemize}
  \item 読み取る環境変数は``QUERY\_{}STRING''ではなく``HTTP\_{}COOKIE''
  \item 変数の区切り文字は``\verb|&|''ではなく``\verb|;|'
\end{itemize}
の2つである。前述のシェルスクリプトの、printf行は環境変数が替わっており、その下にある3つのsed行はCookie変数文字列のフォーマットをCGI変数文字列のフォーマットに変換し、cgi-nameコマンドに流用するために追加したものである。

\subsection*{解説}

CookieのフォーマットについてはRFC 6265で詳しく定義さてれいるが、次のような文字列でやってくる。
\begin{quote}
	\textit{name1}\verb|=|\textit{var1}\verb|; |\textit{name2}\verb|=|\textit{var2}\verb|; ...|
\end{quote}

このルールさえわかれば自力でやるもの難しくはないが、CGI変数文字列とよく似ているので
``cgi-name''コマンドと``nameread''コマンドで処理できるように変換するのが簡単だ。

\subsection*{参照}

\noindent
→RFC 6265文書\footnote{\verb|http://tools.ietf.org/html/rfc6265|} \\
→\ref{recipe:GETmethod}(CGI変数の取得(GETメソッド編)) \\
→\ref{recipe:make_cookie}(シェルスクリプトおばさんの手づくりCookie(書き込み編))
