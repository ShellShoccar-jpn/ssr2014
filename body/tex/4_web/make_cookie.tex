\section{シェルスクリプトおばさんの手づくりCookie(書き込み編)}
\label{recipe:make_cookie}

\subsection*{問題}
\noindent
$\!\!\!\!\!$
\begin{grshfboxit}{160.0mm}
	掲示板Webアプリケーションを作ろうと思う。
	投稿者の名前とe-mailアドレスをCookieで、クライアント(Webブラウザー)に1週間覚えさせたい。
\end{grshfboxit}

\subsection*{回答}
順を追っていけばシェルスクリプトでも手づくり(POSIXの範囲)でCookieが焼ける(Cookieヘッダーを作れる)し、
クライアントから読み取ることもできる。しかしやることがたくさんあるので、POSIXの範囲で実装した``mkcookie''コマンド\footnote{\verb|https://github.com/ShellShoccar-jpn/misc-tools/blob/master/mkcookie|}をダウンロードして使うことにする。

そして例えば、名前(name)とメールアドレス(email)をCookieに覚えさせるCGIスクリプトであれば、次のように書く。
\paragraph{掲示板で名前とメールアドレスをCookieに覚えさせるCGIスクリプト(bbs.cgi)}  \\
\begin{frameboxit}{160.0mm}
\begin{verbatim}
	#! /bin/sh
	
	Tmp=/tmp/${0##*/}.$$                 # 一時ファイルの元となる名称
	
	# (名前とメールアドレスを設定するための何らかの処理)
	
	# "変数名+スペース1文字+値" で表現された元データファイルを作成
	cat <<-FOR_COOKIE > $Tmp-forcookie
	    name $name
	    email $email
	FOR_COOKIE
	
	# Cookie文字列を作成
	cookie_str=$(mkcookie -e +604800 -p /bbs -s Y -h Y $Tmp-forcookie)
	             # -e +604800: 有効期限を604800秒後(1週間後)に設定
	             # -p /bbs   : サイトの/bbsディレクトリー以下で有効なCookieとする
	             # -s Y      : Secureフラグを付けて、SSL接続時以外には読み取れないようにする
	             # -h Y      : httpOnlyフラグを付けて、JavaScriptには拾わせないようにする
	
	# HTTPヘッダーを出力
	cat <<-HTTP_HEADER
	    Content-Type: text/html$cookie_str
	    
	HTTP_HEADER
	
	# (ここでHTMLのボディー部分を出力)
	
	rm -f $Tmp-*                         # 用が済んだら一時ファイルを削除
\end{verbatim}
\end{frameboxit}

mkcookieコマンドに渡す変数は、1変数につき1行で
\begin{quote}
	\textbf{変数名}$<$半角スペース1文字$>$\textbf{値}
\end{quote}
という書式にして作る。変数名と値の間に置く半角スペースは1文字にすること。もし2文字にすると2文字目は値としての半角スペースとみなすので注意。

mkcookieコマンドのオプションについては、``\verb|--help|''オプションなどで表示されるUsageを参照されたい。RFC 6265で定義されている属性に対応しているのですぐわかるだろう。

最後に、ここで出来上がったCookie文字列は、出力しようとしている他のHTTPヘッダーに付加して送る。
注意すべき点が1つある。\textbf{mkcookieコマンドは、先頭に改行を付ける仕様になっている}ので、前述の例のように他のヘッダー(例えばContent-Type)の行末に付加し、単独の行とはしないよう気を付けなければならないということだ。

\subsection*{解説}

クライアントにCookieを送るためにはまず、Cookie文字列がどんな仕様になっているかを知る必要がある。そこで具体例を示そう。まず、次のような条件があるとする。
\begin{itemize}
  \item 投稿者の名前(name)は、「6号さん」
  \item 投稿者のメールアドレス(email)は、``\verb|6go3@example.com|''
  \item 有効期限は、現在(2015/06/01 10:20:30とする)から1週間後
  \item サイトの``/bbs''ディレクトリー以下で有効
  \item ``example.com''というドメインでのみ有効
  \item Secureフラグ(SSLでアクセスしている時のみ)有効
  \item httpOnlyフラグ(JavaScriptには取得させない)有効
\end{itemize}

この時に生成すべきCookie文字列は次のとおりだ。\\
\begin{frameboxit}{160.0mm}
\begin{verbatim}
	Set-Cookie: name=6%E5%8F%B7%E3%81%95%E3%82%93; expires=Tue, 06-Jan-2015 19:20:30 GMT; path=/bbs;
	 domain=example.com; Secure; HttpOnly
	Set-Cookie: email=6go3%40example.com; expires=Tue, 06-Jan-2015 19:20:30 GMT; path=/bbs; domain=e
	xample.com; Secure; HttpOnly
\end{verbatim}
\end{frameboxit}

つまり、``Set-Cookie:''という名前のHTTPヘッダーを用意し、そこに、\textit{変数名}=\textit{値}
\begin{itemize}
  \item \textit{変数名}\verb|=|\textit{値}(必須)
  \item \verb|expires=|\textit{有効期限の日時}(RFC 2616 Sec3.3.1形式、省略可)
  \item \verb|path=|\textit{URL上で使用を許可するディレクトリー}(省略可)
  \item \verb|domain=|\textit{URL上で使用を許可するドメイン}(省略可)
  \item \verb|Secure|(省略可)
  \item \verb|HttpOnly|(省略可)
\end{itemize}
という各種プロパティーを、セミコロン区切りで付けていく。もし送りたいCookie変数が複数ある場合は、1つ1つに``Set-Cookie:''行をつけ、expires以降のプロパティーは同じものを使えばよい。

このようなCookie文字列を生成するにあたっては、ここまで紹介してきたレシピのうちの2つを活用する。値をURLエンコードするには\ref{recipe:URL_encode}(URLエンコードする)、有効日時の計算には\ref{recipe:utconv}(シェルスクリプトで時間計算を一人前にこなす)だ。有効日時は、``RFC 2616 Sec3.3.1形式''ということになっているが、その形式を作るには次のコードで可能だ。\\
\begin{frameboxit}{160.0mm}
\begin{verbatim}
	TZ=UTC+0 date +%Y%m%d%H%M%S |
	TZ=UTC+0 utconv             | # UNIX時間に変換
	awk  '{print $1+86400}'     | # 有効期限を1日としてみた
	TZ=UTC+0 utconv -r          | # UNIX時間から逆変換
	awk '{                      #   "Wdy, DD-Mon-YYYY HH:MM:SS GMT"形式に変換
	  split("Jan Feb Mar Apr May Jun Jul Aug Sep Oct Nov Dec",monthname);
	  split("Sun Mon Tue Wed Thu Fri Sat",weekname);
	  Y = substr($0, 1,4)*1; M = substr($0, 5,2)*1; D = substr($0, 7,2)*1;
	  h = substr($0, 9,2)*1; m = substr($0,11,2)*1; s = substr($0,13,2)*1;
	  Y2 = (M<3) ? Y-1 : Y; M2 = (M<3)? M+12 : M;
	  w = (Y2+int(Y2/4)-int(Y2/100)+int(Y2/400)+int((M2*13+8)/5)+D)%7;
	  printf("%s, %02d-%s-%04d %02d:%02d:%02d GMT\n",
	         weekname[w+1], D, monthname[M], Y, h, m, s);
	}'
\end{verbatim}
\end{frameboxit}

このように、Cookieを手づくりするのも本書のレシピをもってすれば十分可能だ。


\subsection*{参照}

\noindent
→\ref{recipe:URL_encode}(URLエンコードする) \\
→\ref{recipe:utconv}(シェルスクリプトで時間計算を一人前にこなす) \\
→\ref{recipe:read_cookie}(シェルスクリプトおばさんの手づくりCookie(読み込み編)) \\
→RFC 6265文書\footnote{\verb|http://tools.ietf.org/html/rfc6265|} \\
→RFC 2616文書\footnote{\verb|http://tools.ietf.org/html/rfc2616|}
