\section{Base64エンコード・デコードする}
\label{recipe:Base64}

\subsection*{問題}
\noindent
$\!\!\!\!\!$
\begin{grshfboxit}{160.0mm}
	メールを扱うシェルスクリプトを書きたい。
	しかしメールでは、Base64エンコードやデコードをしなければならない場面が多数ある。どうやればいいか?
\end{grshfboxit}

\subsection*{回答}

Linuxあたりだとbase64というコマンドが標準でインストールされていたりするので
それを使う手もあるが、残念ながらPOSIX準拠コマンドではない。
とはいえBase64の変換アルゴリズムはさほど複雑ではなく、
公開されている仕様(RFC 3548)や既存のbase64コマンドに準拠するものをPOSIXの範囲で実装したものがあるのでそれを使う。

\subsection*{Base64コマンドの使い方}

どのようにして実装したのかは解説の項で説明するとして、まずは使ってみよう。

Base64コマンドをPOSIXの範囲で書き直し、GitHubに公開したので、手元の環境になければ下記のページからそれをダウンロードする。

\begin{verbatim}
	https://github.com/ShellShoccar-jpn/misc-tools/blob/master/base64
\end{verbatim}

ダウンロードして実行権限を与えたら、使ってみる。標準入出力経由でデータを渡すとBase64に変換される。
\begin{screen}
	\verb!$ echo 'ShellShoccar' | ./base64! \return    \\
	\verb|U2hlbGxTaG9jY2FyCg==|                        \\
	\verb|$ |
\end{screen}

\verb|-d|オプションを付ければデコードになるので、先程Base64変換した文字列を渡せば元の文字列が出てくる。
\begin{screen}
	\verb!$ echo 'U2hlbGxTaG9jY2FyCg==' | ./base64 -d! \return \\
	\verb|ShellShoccar|                                        \\
	\verb|$ |
\end{screen}

エンコードとデコードは互いに逆変換なので、それらを連続して通せば元の文字列が出力される。
\begin{screen}
	\verb!$ echo 'ShellShoccar' | ./base64 | ./base64 -d! \return \\
	\verb|ShellShoccar|                                           \\
	\verb|$ |
\end{screen}

\subsection*{解説}

Base64の仕様は複雑ではないので素直に実装すればよい、とはいえ、
シェルスクリプトにとっては大きな問題がある。

それはBase64が対象としているデータはテキストデータのみならず
バイナリーデータもあるのだが、シェルスクリプトではNULL文字$<\mathrm{0x00}>$を扱うのが大変なのである。
しかしこの問題は、\ref{recipe:binary}(バイナリーデータを扱う)で克服した。
先に紹介したPOSIX版base64コマンドは、そのレシピで示したコードにBase64の仕様を追加して作られている。

\subsection*{参照}

\noindent
→\ref{recipe:binary}(バイナリーデータを扱う) \\
→\ref{recipe:sendjpmail}(シェルスクリプトでメール送信)
